%
% To start the document, use
%  \levela{...}
% For lover level, sections use
%  \levelb{...}
%  \levelc{...}
%
\levela{Sensor modelling}
 \label{sec:sensor}


%
% Document history, format:
%  \starthistory
%    date1 & text .... \\
%    date2 & text .... \\
%    ....
%  \stophistory
%
\starthistory
  000321 & Created and written by Patrick Eriksson.\\
\stophistory


%
% Symbol table, format:
%  \startsymbols
%    ... & \verb|...| & text ... \\
%    ... & \verb|...| & text ... \\
%    ....
%  \stopsymbols
%
%
\startsymbols
  -- & -- & -- \\
 \label{symtable:sensor}     
\stopsymbols



%
% Introduction
%
A sensor model is needed because a practical instrument consistently
gives spectra deviating from the hypothetical monochromatic pencil
beam spectra provided by the atmsopheric part of the forward model
(i.e. $\y\neq\iv$ always). For a radio (heterodyne) instrument, the
most influential sensor parts are the antenna, the mixer, the sideband
filter and the spectrometer. Limb sounding observations are also
affected by Doppler shifts, but this effect is not considered here, it
is assumed to be treated seperately.




\levelb{Implementation strategy}
 \label{sec:sensor:strategy}
 
 The modelling of a sensor part is either a summation of different
 frequency components (mixer), or a weighting of the spectra as a
 function of frequency (spectrometer) or viewing direction (antenna)
 with the instrument response of concern. In all cases it is
 possible to describe the sensor influence by an analytical
 expression. See e.g. \citet{eriksson:97a} for more details.
 These analytical expressions can be implemented and solved for each
 run of the sensor model, but this would be relatively computationally
 demanding for cases when the settings are kept constant, as the
 calculations are duplicated in unnecessary manner, and we want to
 find a better implementation strategy.
 
 Summation and weighting of the spectral components are both linear
 operations, and thus it is possible to model the effect of the
 different sensor parts as subsequent matrix multilications of the
 monochromatic pencil beam spectrum, as suggested in \citet{eriksson:00a}:
 \begin{eqnarray}
   \y = \Hm_n\dots\Hm_2\Hm_1\iv + \merr
 \end{eqnarray}
 where $n$ is the number of sensor parts to consider, and this results
 in that the sensor model can be expressed as a single matrix
 multiplication (Eq. \ref{eq:formalism:H})
 \begin{eqnarray}
   \y = \Hm\iv + \merr                     \nonumber
 \end{eqnarray}
 Applying Equation \ref{eq:formalism:H} for the sensor model will
 clearly give very rapid calculations, and we must find ways to
 calculate $\Hm$.



\levelb{Integration as vector multiplication} 
 \label{sec:sensor:integr}
  
 The effect of both the antenna and the spectrometer can be expressed
 as an integral \citep[e.g.][Eq. 86 and 94]{eriksson:97a}, and the
 question is how to transform these integrals into matrix operations.
  
 The problem at hand is that the antenna and spectrometer responses
 and the viewing angle and frequency grids are known, while the spectral
 values are unknown. This problem corresponds to determine a (row)
 vector $\mat{h}$ that multiplied with an unknown (column) vector,
 $\mat{g}$, approximates the integral of the product between the
 functions $g$ and $f$:
 \begin{equation} 
   \mat{hg} = \int{f(x)g(x) \dd x}
   \label{eq:sensor:integral_problem}
 \end{equation}
 where $\mat{g}$ contains values of $g$ at some discrete points. The
 functions $f$ is here the response for some sensor part, and $g$
 holds the spectral values.
 
 The shape of $f$ and $g$ between the grid points must be known to
 solve this problem. Here it will be assumed that both functions are
 piecewise linear.
  
 Following Figure \ref{fig:sensor:vecintegr}, the function $g$ can between
 the points $x_1$ and $x_4$ be expressed as a sum of the two unknown
 values $g_1$ and $g_2$:
 \begin{equation}
   g(x) = g_1 + (g_2-g_1)\frac{x-x_1}{x_4-x_1} =
           g_1 \frac{x_4-x}{x_4-x_1} + g_2\frac{x-x_1}{x_4-x_1}
 \end{equation}
 which can be rewritten as
 \begin{equation}
   g(x) = g_1(a+bx)+g_2(c-bx), \qquad x_1 \leq x \leq x_4
 \end{equation}
 where
 \begin{eqnarray}
    a=\frac{x_4}{x_4-x_1}, \qquad b=\frac{-1}{x_4-x_1}, \qquad 
    c=\frac{-x_1}{x_4-x_1}   \nonumber
 \end{eqnarray}
 
 A shorter expression can be obtained for the function $f$ as the
 values $f_1$ and $f_2$ are known:
 \begin{equation}
   f(x) = (d+ex), \qquad x_2 \leq x \leq x_3
 \end{equation}
 where 
 \begin{eqnarray}
    d=f_1-x_2\frac{f_2-f1}{x_3-x_2} \qquad e=\frac{f_2-f_1}{x_3-x_2} \nonumber
 \end{eqnarray}

 \begin{figure}[tb]
    \begin{center}
      \includegraphics*{Figs/vecintegr.eps}
      \caption{The quantities used in Section \ref{sec:sensor:integr}.}  
      \label{fig:sensor:vecintegr} 
    \end{center} 
 \end{figure}
  
 The integral in Equation \ref{eq:sensor:integral_problem} can now for
 ranges between $x_2$ and $x_3$ be calculated analytically in a
 straightforward manner:
 \begin{eqnarray}
    \int_{x_a}^{x_b}{f(x)g(x) \dd x} =
    \int_{x_a}^{x_b}{\big(d+ex\big)\big(g_1(a+bx)+g_2(c-bx)\big) \dd x}  
    =\dots= \nonumber\\
    \bigg[ g_1x\Big(ad+\frac{x}{2}(bd+ae)+\frac{x^2}{3}be\Big) + 
           g_2x\Big(cd+\frac{x}{2}(ce-bd)-\frac{x^2}{3}be \Big)
           \bigg]_{x_a}^{x_b}
    \label{eq:sensor:integr_weights}
 \end{eqnarray}
 For the practical calculations, the integral is solved from one grid
 point to next, of either $\mat{f}$ or $\mat{g}$.  For the case
 shown in Figure \ref{fig:sensor:vecintegr}, the integration order would be
 $(x_a,x_b)=(x_2,x_3)$, $(x_a,x_b)=(x_3,x_4)$, $(x_a,x_b)=(x_4,x_5)$
 \ldots\ The integration starts at $x_2$ as the functions are assumed
 to be zero outside their defined ranges (i.e. $f=0$ for $x<x_2$).
  
 Using Equation \ref{eq:sensor:integr_weights}, we can now determine how to
 calculate $\mat{h}$. For each integration step, $\mat{h}_i$ and
 $\mat{h_{i+1}}$ are increased as
 \begin{eqnarray}
    \mat{h}_i \!\! &=& \!\! \mat{h}_i +    
              x_b\Big(ad+\frac{x_b}{2}(bd+ae)+\frac{x_b^2}{3}be\Big) - 
              x_a\Big(ad+\frac{x_a}{2}(bd+ae)+\frac{x_a^2}{3}be\Big) 
    \nonumber \\
    \mat{h}_{i+1} \!\! &=& \!\! \mat{h}_{i+1} +
              x_b\Big(cd+\frac{x_b}{2}(ce-bd)-\frac{x_b^2}{3}be\Big) - 
              x_a\Big(cd+\frac{x_a}{2}(ce-bd)-\frac{x_a^2}{3}be\Big) 
    \nonumber
 \end{eqnarray}
 where $i$ is the index for which $\mat{x}^i \leq x_a$ and $x_b \leq
 \mat{x}^{i+1}$. The vector $\mat{h}$ is initialised with
 zeros before the integration starts.



\levelb{Summation as vector multiplication}
  
 The influence of the mixer and sideband filter of the sensor
 correspond to a summation of pairs of frequency components. The two
 frequencies of the pair are related as
 \begin{equation}
    \f' = 2\f_{LO}-\f
 \end{equation}
 where $\f_{LO}$ is the frequence of the local oscillator signal, and
 $\f'$ is denoted as the image frequency.

 \begin{figure}[tb]
  \begin{center}
    \includegraphics*[width=0.9\hsize]{Figs/sideband.eps}
    \caption{Schematic description of image frequency and sideband filtering.}
   \label{fig:sensor:sideband} 
  \end{center} 
 \end{figure}
 
 The intensity correspondance after the mixer and the sideband filter
 can be written as
 \begin{equation}
   I_{IF}(\f) = \frac{f_s(\f)I(\f)+f_s(\f')I(\f')}{f_s(\f)+f_s(\f')}
  \label{eq:sensor:sband}
 \end{equation}
 where $I(\f)$ is the intensity for frequency $\f$ and $f_s$ the response
 of the sideband filter as a function of frequency.

 If the intensity is assumed to vary linearly between the points of the
 frequency grid, Equation \ref{eq:sensor:sband} can be written as
 \begin{eqnarray}
   I_{IF}(\f^i) &=& \frac{1}{f_s(\f_i)+f_s(\f_i')}\Big[ f_s(\f_i)I(\f_i)+ \nonumber \\ 
      & & + \frac{f_s(\f_i')}{\f_{j+1}-\f_j} ( I(\f_j)(\f_{j+1}-\f_i')
           + I(\f_{j+1})(\f_i'-\f_j) )  \Big]
  \label{eq:sensor:mixer}
 \end{eqnarray}
 where $f_s$ for the different frequencies is obtained by an analytical
 expression or interpolation, and $\f_j$ and $\f_{j+1}$ are the two points
 of the frequency grid surrounding the image frequency, $\f_i'$. The row
 of the $\Hm$ matrix corresponding to $\f^i$ is then
 \begin{eqnarray}
    \mat{h}^i &=& \frac{f_s(\f_i)}{f_s(\f_i)+f_s(\f_i')}    \nonumber \\
    \mat{h}^j &=& \frac{f_s(\f_i')}{f_s(\f_i)+f_s(\f_i')}
                  \frac{\f_{j+1}-\f_i'}{\f_{j+1}-\f_j}     \nonumber \\
    \mat{h}^{j+1} &=& \frac{f_s(\f_i')}{f_s(\f_i)+f_s(\f_i')}
                  \frac{\f_i'-\f_j}{\f_{j+1}-\f_j}     \nonumber
 \end{eqnarray}
 where $\mat{h}^i$ is the value of $\mat{h}$ for frequency $\f_i$ etc.
 Remaining values of $\Hm$ are zero.

 For the special case when the image frequency matchs perfectly a frequency
 grid point, the equations above can be simplified to give
 \begin{eqnarray}
    \mat{h}^i &=& \frac{f_s(\f_i)}{f_s(\f_i)+f_s(\f_i')}    \nonumber \\
    \mat{h}^j &=& \frac{f_s(\f_i')}{f_s(\f_i)+f_s(\f_i')}    \nonumber
 \end{eqnarray}
 
 The frequency grid after the mixer has a point for each unique
 frequency before the mixer, i.e. frequencies of the image band are
 mirrored to thier corresponding frequencies in the primary band. This
 fact results in that the mixer-sideband transfer matrix is square if
 there are no perfect matches between primary and image frequencies.
 




%%% Local Variables: 
%%% mode: latex 
%%% TeX-master: "main" 
%%% End:

