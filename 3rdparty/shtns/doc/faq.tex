\documentclass[10pt,a4paper]{article}
\usepackage[utf8]{inputenc}
\usepackage[T1]{fontenc}
\usepackage{amsmath}
\usepackage{amsfonts}
\usepackage{amssymb}
\usepackage{graphicx}
\begin{document}
	\title{Frequently asked questions}
	
	\section{FAQ 01: Noisy divergence-free velocity field despite}
	Under special circumstances, initialization of a velocity field in grid space can lead to artificial oscillations with the spectral transformation.
	Such a velocity field can be found in \cite{nair2010} and is e.g.\,given by
	\begin{eqnarray}
		u(\lambda,\theta) &=& \sin^2(\lambda/2) \sin(2\theta)\\
		v(\lambda,\theta) &=& 1/2 \sin(\lambda) \cos(\theta)\\
		\psi(\lambda,\theta) &=& \sin^2(\lambda/2) \cos^2(\theta).
	\end{eqnarray}
	Setting this up in velocity space $(u,v)^T$ field results in a non-divergence free field.
This particular field of Case-1 is not smooth at the poles (not infinitely differentiable).
This means that it is  not bandwidth limited: it can be described accurately only by an infinite expansion of SH.
In this case, an SH truncation breaks the divergence freeness.
It is as if the divergence freeness is ensured by ever adding an addition
An example is provided in folder \texttt{faq\_01\_divergence\_freeness}.
	

\begin{thebibliography}{9}
	\bibitem{nair2010}{Nair, R. D., \& Lauritzen, P. H. (2010). A class of deformational flow test cases for linear transport problems on the sphere. Journal of Computational Physics, 229(23), 8868–8887. https://doi.org/10.1016/j.jcp.2010.08.014}
\end{thebibliography}

\end{document}