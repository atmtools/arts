%
% To start the document, use
%  \chapter{...}
% For lover level, sections use
%  \section{...}
%  \subsection{...}
%
\chapter{Introduction}
 \label{sec:intro}

%
% Document history, format:
%  \starthistory
%    date1 & text .... \\
%    date2 & text .... \\
%    ....
%  \stophistory
%
\starthistory
  02xxxx & xxx.\\
\stophistory


Some nice welcome text ...

\section{Temporary internal notes}

Below you find a list of things to do. Please add and remove items
when appropriate. Let's try to keep this list rather complete and to
have a person assigned for each point. This in order to keep up the
speed of creating this user guide and that no point is continously
expected to be fixed by someone else.
\begin{itemize}
\item Write this section. [Stefan/Patrick]
\item Section 2 is marked as under construction. Correct? Update or
  remove comment. [Stefan]
\item Something shall be written about polarisation and Stokes in Section
   3.3. [Christian/Claudia]
\item Introduction to absorption and refractive index has to be written.
  The corresponding chapter in Part 1 is empty.
  [???]
\item Section 3.5.3 on Sensor polarisation to be written. Finer details
  shall be put in a chapter on Sensor characteristics in Part 1. 
  [Patrick/Christian]
\item The theory we use for radiative transfer shall be described in Part
  IV in as general terms as possible. The start of the scattering chapter
  should be moved to the new chapter. The case without scattering can then
  be described as a special case of the general expression. To gather the
  description of radiative transfer should be better than to describe it 
  in parts at different places in the user guide. [Claudia/Christian]
\item An introductory part on scattering calculations shall be found
   in Sec. 3.8 [Claudia/Shreerekha]
\item Start a chapter on clear sky radiative transfer for Part I. [Patrick]
\item Is scattering chapter OK? [Claudia/Shreerekha]
\item Fix chapter on agendas. [Stefan]
\item There are some FIXME in chapter on Polarisation and Stokes Parameters.
  [Christian]
\item In the appendix for WSM: The method names appear strange (starts
  with levelb). The list of input and output varaibles needs a row break.
  [Oliver]
\end{itemize}




\section{Documentation guide}
%====================
\label{sec:intro:guide}

Describe where different type of information can be found. For
example, refer to full control file examples in \verb|doc/examples|.





\section{Background}
%====================
\label{sec:intro:background}

The number of satellite sensors in the millimeter and sub-millimeter
spectral range is rapidly growing. They use various frequency
bands and observation geometries. Two important groups of
sensors are for example the nadir viewing millimeter wave
sensors like AMSU\footnote{The \textbf{A}dvanced
  \textbf{M}icrowave \textbf{S}ounding \textbf{U}nit is a
  sensor on board the polar orbiting satellites of the
  US-American National Aeronautics and Space Administration.}
and the limb viewing sub-millimeter wave sensors like the
planned SMILES\footnote{The \textbf{S}uperconducting
  Sub-\textbf{Mi}llimeter Wave \textbf{L}imb \textbf{E}mission
  \textbf{S}ounder is a Japanese Sensor which will be flown
  for the first time on the International Space Station.}.

For the data analysis all such sensors require accurate and
fast forward models, which can simulate measurements for a
given atmospheric (and maybe ground) state. Depending on the
objective of the sensor, the measurement will depend for
example on the distribution of atmospheric temperature, water
vapor, ozone, and many other trace gases.

So far, a lot of effort has been wasted in developing dedicated
forward models for different sensors, although all these models have
many features in common. Moreover, existing models were not easily
modifiable and extendable. Hence, it was decided to develop a new
model which emphasizes modularity, extensibility, and generality.

[* Describe how ARTS was initiated and started. Release of version 1. *]


\section{What is ARTS}
%====================
\label{sec:intro:whatis}

[* ??? *]


\section{The scope of ARTS}
%====================
\label{sec:intro:scope}

[* Update old text and add new stuff. *]

%The present version of ARTS is limited to cases where scattering can
%be neglected and local thermodynamic equilibrium applies. ARTS has
%been developed having passive emission measurements in mind, put pure
%transmission (that is, the emission is neglected) observations are
%also handled. The forward model can be used to simulate measurements
%for all (normal?)  observation geometries: ground-based, nadir
%looking, limb sounding and balloon/aircraft measurements. It can be
%noted that ARTS handles measurements from a point inside the
%atmosphere, such as an aircraft or a balloon, in a downward direction.
%ARTS covers so far only monochromatic pencil beam calculations, that
%is, no sensor characteristics can be included. This part is presently
%covered by the AMI (ARTS Matlab interface) set of Matlab functions 
%(see below). Sensor characteristics will be included in ARTS.

%Beside providing set of spectra, ARTS calculates weighting functions
%for a number of variables. Analytical expressions for the weighting
%functions are used for species, continuum absorption and ground
%emission, and for temperature if hydrostatic equilibrium is \emph{not}
%assumed. Weighting functions are also provided for pointing off-sets,
%calibration and temperature (with hydrostatic equilibrium).



\section{Additional tools}
%====================
\label{sec:intro:tools}

[* Update old text and add new stuff. *]

%For Matlab users there are two accompanying packages called AMI and
%Qpack\footnote{AMI is distributed by ARTS, while Qpack is a separate
%  package} which extends the usage of ARTS considerably. First of all,
%AMI has functions to include sensor characteristics in the
%calculations. AMI has further functions to read and write ARTS data
%file, and various functions that are of general usage. Qpack is an
%Matlab environment to perform OEM inversions and producing set of
%spectra to test the inversions, where ARTS is used as calculating
%engine.



%%% Local Variables: 
%%% mode: latex
%%% TeX-master: "uguide"
%%% End: 
