\chapter{Weighting functions}
 \label{sec:wfuns}

 \starthistory
 11xxyy & First complete version by Patrick Eriksson.\\
 \stophistory

\graphicspath{{Figs/wfuns/}}


Inversions of both OEM and Tikhonov type require that the ``weighting
functions'' can be provided by the forward model \citep[see
e.g.][]{eriksson:analy:00}. A retrieval characterisation following
\citet{rodgers:90} raises the same demand. A weighting function is defined as 
\begin{equation}
  \frac{\partial \MsrVct}{\partial x_i}
  \label{eq:wfuns:ki}
\end{equation}
where \MsrVct\ is the vector of measurement data and $x_i$ is one forward model
(scalar) variable. 

A few words about the nomenclature, a weighting function forms a column of the
complete weighting function matrix, \aWfnMtr (see \theory,
Sec.~\ref{T-sec:formalism:wfuns}). A probably more commonly used name for
\aWfnMtr\ is the Jacobian, and in the documentation of ARTS you find both
terms. Also the term ``the Jacobians'' is used, which shall be interpreted as
``the weighting functions''. These names refer normally to the partial
derivates with respect to the variables to be retrieved, the state vector
\SttVct. However, in the context of retrieval characterisation, the same matrix
for the remaining model paramaters is of equally high interest, denoted as
\aWfnMtr{\FrwMdlVct}\ in Sec.~\ref{T-sec:formalism:wfuns} of \theory. In the
same manner, the terms inversion and retrieval are used interchangeably.

The main task of the user is to select which quantities that shall be
retrieved, and to define the associated retrieval grids. These aspects must be
considered for successful inversions, but are out of scope for this document.
Beside for the most simple retrievals, it is further important to understand
how the different weighting functions are calculated. A practical point is the
calculation speed, primarily determined if perturbations or analytical
expressions are used (Sec.~\ref{sec:wfuns:overw}). The derivation of the
weighting functions involves also some approximations (due to theoretical or
practical considerations). Such approximations can (and must) be accepted, but
will result in a slower convergence (the inversion will require more
iterations). Due to these later aspects, and to meet the needs of more
experienced users, this section is relatively detailed and contains a (high?)
number of equations.



\section{Overwiev}
%==============================================================================
\label{sec:wfuns:overw}

There are two main approaches for calculation the weighting functions, by
analytical expressions and by perturbations. We start with the conceptually
simplest one, but also the more inefficient aprroach.


\subsection{Perturbations}
%==============================================================================
\label{sec:wfuns:pert}

