\levela{Sensor modeling}
 \label{sec:sensor}


%
% Document history, format:
%  \starthistory
%    date1 & text .... \\
%    date2 & text .... \\
%    ....
%  \stophistory
%
\starthistory
  031205 & Revised and extended by Mattias Ekstr\"om. \\
  000826 & Written for ARTS-1 by Patrick Eriksson.\\
\stophistory


%
% Symbol table, format:
%  \startsymbols
%    ... & \verb|...| & text ... \\
%    ... & \verb|...| & text ... \\
%    ....
%  \stopsymbols
%
%
\startsymbols
  & \artsstyle{antenna\_diagram} & The antenna pattern \\
  & \artsstyle{antenna\_dim} & The dimensionality of the antenna pattern \\
  & \artsstyle{f\_backend} & The frequency grid of the backend channels \\
  & \artsstyle{f\_mixer} & Output frequency grid from the mixer \\
  & \artsstyle{lo} & Local oscillator frequency \\
  & \artsstyle{sensor\_pol} & The sensor polarisation response \\
  & \artsstyle{sensor\_response} & The total sensor forward matrix \\
  & \artsstyle{sensor\_response\_f} & The sensor frequency grid \\
  & \artsstyle{sensor\_response\_za} & The sensor zenith angle grid \\
  & \artsstyle{sensor\_response\_aa} & The sensor azimuth angle grid \\
  & \artsstyle{sensor\_pol} & The sensor polarisation response \\
  & \artsstyle{sensor\_response\_za} & The sensor zenith angle grid \\
  & \artsstyle{sensor\_rot} & The rotation of the sensor between measurement blocks \\
 \label{symtable:sensor}     
\stopsymbols



%
% Introduction
%

A sensor model is needed because a practical instrument gives
consistently spectra deviating from the hypothetical monochromatic
pencil beam spectra provided by the atmospheric part of the forward
model (that is $\mathbf{y} \neq \mathbf{i}$ always). For a radio (heterodyne)
instrument, the most influential sensor parts are the antenna, the
mixer, the sideband filter and the spectrometer. Limb sounding
observations are also affected by Doppler shifts, but this effect is
not considered here, it is assumed to be treated separately.

In the follow text we will use the terms sensor to denote the total
sensor configuration, i.e. the the whole object that will be
represented by the WSV \artsstyle{sensor\_response}. To denote 
individual parts of the sensor, such as the antenna or the mixer, we
use the terms sensor parts or instrument, especially when taking
of characteristics.
%Conversion of radiances to brightness temperatures is also treated
%here.

\levelb{Theory}

\levelb{Concepts in ARTS}
In an effort to keep a consitency between the different sensor parts, 
some general concepts has been determined for the variables and 
methods connected to sensor modelling.
Examples of this are the way the instrument characteristics are 
stored and the of course the way the sensor response is modelled 
as matrix, as decribed above (REF?).

\levelc{Sensor response matrix?}

\levelc{Instrument characteristics}
To be able of computing the response from a certain sensor part, we 
need the instrument characteristics. This information thus have to 
be stored in such a way that we can incorporate it in our forward 
model. ARTS uses XML-files to store and retrieve variables from 
files. Thus it is easiest to store the characteristics in form of 
XML-files, but there exist also methods for computing simple gaussian
characteristics. See section~\ref{sec:sensor:WSM} for more details.

Even though the different sensor parts are very different in their
representation of their characteristics, we have tried to apply the 
same concept when applying them to the forward model. This is where
the concept of \textit{single or complete} is used. The words 
\textit{single} and \textit{complete} describes the way the given 
characteristics are applied. This might better be understood with an
example, e.g. the antenna, if only one set of antenna characteristics
are given the same set will for instance be used for all viewing
directions. If on the other hand a individual set of antenna 
characteristics are given for each viewing direction, then that
individual set will be used for each individual view direction.
What this imply is that you can't give instrument characteristics
for only half of the cases, you have to give one set, \textit{single},
or a set for each case, \textit{complete}.

How this is implemented for each sensor part will be described below.

\leveld{Antenna 1D characteristics}
For the antenna characteristics we have implemented a separate WSV 
\artsstyle{antenna\_diagram}. This variable is of the type 
\typeindex{ArrayOfArrayOfMatrix} to accomplish the diversity antenna
configurations and characterisations. For example this variable can 
be used to describe a multi-antenna setup with different viewing 
directions.

To explain the definition of \artsstyle{antenna\_diagram}, we will go through it backwards, starting with the \artsstyle{Matrix}. The \textit{single or complete} concept applies to this matrix in the way that it describes the angle dependence of the antenna characteristics for one or several frequencies. The basic concept of the matrix is that the first column describes a relative grid of angles and the following column(s) contain the antenna diagram described for the angles given in the first column. Thus if the matrix contains two columns, one with the angular grid and one with weights, the same weights will be used for all frequencies. The other possibility is to provide one column of weights for each frequency, so the total number of columns will be equal the number of frequencies plus one. In the case of a complete description, the order of the columns follows the order of the frequencies.

The next level in the \artsstyle{antenna\_diagram} is then the \typeindex{ArrayOfMatrix}. The array contains the above described matrices, and the reason to stack them into an array is that each element in the array corresponds to different polarisations given by the rows of \artsstyle{sensor\_pol}. Here the a single description implies that the array only contains one matrix, which will be used for all polarisation. The other option is that the array contains equal amount of matrices as polarsations.

The highest level is then the \artsstyle{ArrayOfArrayOfMatrix}, where each array of arrays corresponds to different viewing angles. In this case with a 1D antenna, this means different zenith angles given as generic input to the WSM \artsstyle{sensor\_responseAntenna1D}. As with all instrument characteristics, this array should either contain one \artsstyle{ArrayOfMatrix} or as many as there are zenith angles given to the WSM.

One could think of the \artsstyle{antenna\_diagram} as a 4D tensor, with dimensions as relative angles, frequency, polarisations and zenith angles. But the reason not to store the characteristics as a tensor is that we want to keep the possibility to include antenna diagrams of different sizes, i.e. relative angular grids with different coarseness. The same goes for the polarisations, for each zenith angle the user have the possibility to give a single or complete description of the polarisation dependence.

\leveld{Mixer and sideband filter charcteristics}


\levelb{Workspace variables}

\levelb{Workspace methods}
\label{sec:sensor:WSM}

\levelb{Controlfile implementation(flow, step-by-step)}



%\levelb{Implementation strategy}
% \label{sec:sensor:strategy}

%\levelc{The sensor transfer matrix}
% \label{sec:sensor:strategy:h}
 
% The modeling of a sensor part is either a summation of different
% frequency components (mixer), or a weighting of the spectra as a
% function of frequency (spectrometer) or viewing direction (antenna)
% with the instrument response of concern. In all cases it is
% possible to describe the sensor influence by an analytical
% expression. See for example \citet{eriksson:97a} for more details.
% These analytical expressions can be implemented and solved for each
% run of the sensor model, but this would be relatively computationally
% demanding for cases when the settings are kept constant, as the
% calculations are duplicated in an unnecessary manner, and we want to
% find a better implementation strategy.
 
% Summation and weighting of the spectral components are both linear
% operations, and thus it is possible to model the effect of the
% different sensor parts as subsequent matrix multiplications of the
% monochromatic pencil beam spectrum, as suggested in \citet{eriksson:00a}:
% \begin{eqnarray}
%   \y = \Hm_n\dots\Hm_2\Hm_1\iv + \merr
% \end{eqnarray}
% where $n$ is the number of sensor parts to consider, and this results
% in that the sensor model can be expressed as a single matrix
% multiplication (Eq. \ref{eq:formalism:H})
% \begin{eqnarray}
%   \y = \Hm\iv + \merr                     \nonumber
% \end{eqnarray}
% Applying Equation \ref{eq:formalism:H} for the sensor model will
% clearly give very rapid calculations, and we must find ways to
% calculate $\Hm$.


%\levelc{Normalization of \Hm}
% \label{sec:sensor:strategy:norm}
 
% It is important that the transfer matrix for
% each sensor part is normalized in such way that a unit response is
% obtained. A unit response signifies here that a constant intensity
% (as a function of frequency or zenith angle) is preserved, that is
% \begin{equation}
%   \mat{u}_2 = \Hm\mat{u}_1
% \end{equation}
% where $\mat{u}_1$ and $\mat{u}_2$ are vectors of appropriate length
% where each element is $1$. This criterion equals that the sum of 
% the elements of each row of \Hm\ is 1.


%\levelb{Integration as vector multiplication} 
% \label{sec:sensor:integr}
  
% The effect of both the antenna and the spectrometer can be expressed
% as an integral \citep[e.g.][Eq. 86 and 94]{eriksson:97a}, and the
% question is how to transform these integrals into matrix operations.
  
% The problem at hand is that the antenna and spectrometer responses
% and the zenith angle and frequency grids are known, while the spectral
% values are unknown. This problem corresponds to determine a (row)
% vector $\mat{h}$ that multiplied with an unknown (column) vector,
% $\mat{g}$, approximates the integral of the product between the
% functions $g$ and $f$:
% \begin{equation} 
%   \mat{hg} = \int{f(x)g(x) \dd x}
%   \label{eq:sensor:integral_problem}
% \end{equation}
% where $\mat{g}$ contains values of $g$ at some discrete points. The
% functions $f$ is here the response for some sensor part, and $g$
% holds the spectral values. The shape of $f$ and $g$ between the grid
% points must be known to solve this problem.


%\levelc{Piecewise linear functions} 
% \label{sec:sensor:integr:lins}
 
% In this section the problem of
% Equation~\ref{eq:sensor:integral_problem} is solved analytically when
% both functions are piecewise linear. The practical solution used
% Qpack is discussed in next section.
  
% Following Figure \ref{fig:sensor:vecintegr}, the function $g$ can between
% the points $x_1$ and $x_4$ be expressed as a sum of the two unknown
% values $g_1$ and $g_2$:
% \begin{equation}
%   g(x) = g_1 + (g_2-g_1)\frac{x-x_1}{x_4-x_1} =
%           g_1 \frac{x_4-x}{x_4-x_1} + g_2\frac{x-x_1}{x_4-x_1}
% \end{equation}
% which can be rewritten as
% \begin{equation}
%   g(x) = g_1(a+bx)+g_2(c-bx), \qquad x_1 \leq x \leq x_4
%   \label{eq:sensor:glin}
% \end{equation}
% where
% \begin{eqnarray}
%    a=\frac{x_4}{x_4-x_1}, \qquad b=\frac{-1}{x_4-x_1}, \qquad 
%    c=\frac{-x_1}{x_4-x_1}   \nonumber
% \end{eqnarray} 
% A shorter expression can be obtained for the function $f$ as the
% values $f_1$ and $f_2$ are known:
% \begin{equation}
%   f(x) = (d+ex), \qquad x_2 \leq x \leq x_3
% \end{equation}
% where 
% \begin{eqnarray}
%    d=f_1-x_2\frac{f_2-f1}{x_3-x_2} \qquad e=\frac{f_2-f_1}{x_3-x_2} \nonumber
% \end{eqnarray}
% \begin{figure}[tb]
%    \begin{center}
%      \includegraphics*{Figs/vecintegr}
%      \caption{The quantities used in Section \ref{sec:sensor:integr}.}  
%      \label{fig:sensor:vecintegr} 
%    \end{center} 
% \end{figure}
% The integral in Equation \ref{eq:sensor:integral_problem} can now for
% ranges between $x_2$ and $x_3$ be calculated analytically in a
% straightforward manner:
% \begin{eqnarray}
%    \int_{x_a}^{x_b}{f(x)g(x) \dd x} =
%    \int_{x_a}^{x_b}{\big(d+ex\big)\big(g_1(a+bx)+g_2(c-bx)\big) \dd x}  
%    =\dots= \nonumber\\
%    \bigg[ g_1x\Big(ad+\frac{x}{2}(bd+ae)+\frac{x^2}{3}be\Big) + 
%           g_2x\Big(cd+\frac{x}{2}(ce-bd)-\frac{x^2}{3}be \Big)
%           \bigg]_{x_a}^{x_b}
%    \label{eq:sensor:integr_weights}
% \end{eqnarray}
% For the practical calculations, the integral is solved from one grid
% point to next, of either $\mat{f}$ or $\mat{g}$. The functions are 
% assumed to be zero outside their defined ranges (for example, $f=0$ 
% for $x<x_2$).
% For the case
% shown in Figure \ref{fig:sensor:vecintegr}, the integration order would be
% $(x_a,x_b)=(x_2,x_3)$, $(x_a,x_b)=(x_3,x_4)$, $(x_a,x_b)=(x_4,x_5)$
% \ldots\
  
% Using Equation \ref{eq:sensor:integr_weights}, we can now determine how to
% calculate $\mat{h}$. For each integration step, $\mat{h}_i$ and
% $\mat{h_{i+1}}$ are increased as
% \begin{eqnarray}
%    \mat{h}_i \!\! &=& \!\! \mat{h}_i +    
%              x_b\Big(ad+\frac{x_b}{2}(bd+ae)+\frac{x_b^2}{3}be\Big) - 
%              x_a\Big(ad+\frac{x_a}{2}(bd+ae)+\frac{x_a^2}{3}be\Big) 
%    \nonumber \\
%    \mat{h}_{i+1} \!\! &=& \!\! \mat{h}_{i+1} +
%              x_b\Big(cd+\frac{x_b}{2}(ce-bd)-\frac{x_b^2}{3}be\Big) - 
%              x_a\Big(cd+\frac{x_a}{2}(ce-bd)-\frac{x_a^2}{3}be\Big) 
%    \nonumber
% \end{eqnarray}
% where $i$ is the index for which $\mat{x}^i \leq x_a$ and $x_b \leq
% \mat{x}^{i+1}$. The vector $\mat{h}$ is initialized with
% zeros before the calculation starts.


%\levelc{Practical solution} 
% \label{sec:sensor:integr:practical}
 
% The functions $f$ and $g$ can in Qpack be treated to be piecewice
% linear or cubic functions. The polynomial order of the two functions
% is set individually. When a function is assumed to be piecewise
% cubic, two points on each side of the range of interest (that is, in
% total 4 points) are used to determine the polynomial. For the end
% ranges, a quadratic polynomial is used as there exists only a single
% point on one of the sides. 
 
% Accordingly, Equation~\ref{eq:sensor:integral_problem} must be
% handled in Qpack for combinations of piecewise linear, quadratic and
% cubic functions. Instead of repeating the calculations in Section
% \ref{sec:sensor:integr:lins} for all possible polynomial
% combinations, a more general solution was implemented. The polynomial
% coefficents for $f$ are simply obtained by doing a polynomial fit to
% the considered points (by the Matlab function \verb|polyfit|). The
% polynomial basis for $g$ ($a$, $b$ and $c$ in Equation
% \ref{eq:sensor:glin}) is obtained by Lagrange's formula (Equation
% \ref{eq:wfuns:lagrange}), which expresses the polynomial that passes
% a fixed set of points. The Lagrange's formula can be written as:
% \begin{eqnarray}
%  g(x) &=& (a_{11}+a_{12}x+\dots+a_{1N}x^N)*g_1 + \nonumber \\
%       & & (a_{21}+a_{22}x+\dots+a_{2N}x^N)*g_2 + \nonumber \\
%       & & \dots \nonumber \\
%       & & (a_{N1}+a_{N2}x+\dots+a_{NN}x^N)*g_N 
%  \label{eq:sensor:pbasis}
% \end{eqnarray}
% With the obtained coefficients for $f$ and $g$, Equation
% \ref{eq:sensor:integr_weights} can easily be solved analytically in a
% general manner. The polynomial pasis is determined by the AMI
% function \verb|pbasis|, the both set of coefficients are
% multiplicated in the function \verb|pbasis_x_pol| and the integral is
% solved by the function \verb|pbasis_integrate|.


%\levelb{Summation as vector multiplication}
% \label{sec:sensor:mixer}
  
% The influence of the mixer and sideband filter of the sensor
% correspond to a summation of pairs of frequency components. The two
% frequencies of the pair are related as
% \begin{equation}
%    \f' = 2\f_{LO}-\f
% \end{equation}
% where $\f_{LO}$ is the frequence of the local oscillator signal, and
% $\f'$ is denoted as the image frequency.

% \begin{figure}[tb]
%  \begin{center}
%    \includegraphics*[width=0.8\hsize]{Figs/sideband}
%    \caption{Schematic description of image frequency and sideband filtering.}
%   \label{fig:sensor:sideband} 
%  \end{center} 
% \end{figure}
 
% The intensity correspondence after the mixer and the sideband filter
% can be written as
% \begin{equation}
%   I_{IF}(\f) = \frac{f_s(\f)I(\f)+f_s(\f')I(\f')}{f_s(\f)+f_s(\f')}
%  \label{eq:sensor:sband}
% \end{equation}
% where $I(\f)$ is the intensity for frequency $\f$ and $f_s$ the response
% of the sideband filter as a function of frequency.

% The frequency grid after the mixer consists of the frequencies inside
% the primary band of the grid before the mixer. To include frequencies
% from the image band (mirrored to the primary band) would need an 
% interpolation in the primary band that could cause unexpected effects.  


%\levelc{Piecewise linear functions} 
% \label{sec:sensor:mixer:lins}

% If the intensity is assumed to vary linearly between the points of the
% frequency grid, Equation \ref{eq:sensor:sband} can be written as
% \begin{eqnarray}
%   I_{IF}(\f^i) &=& \frac{1}{f_s(\f_i)+f_s(\f_i')} \bigg[ f_s(\f_i)I(\f_i)+ \nonumber \\ 
%      & & + \frac{f_s(\f_i')}{\f_{j+1}-\f_j} \Big( I(\f_j)(\f_{j+1}-\f_i')
%           + I(\f_{j+1})(\f_i'-\f_j) \Big)  \bigg]
%  \label{eq:sensor:mixer}
% \end{eqnarray}
% where $f_s$ for the different frequencies is obtained by linear
% interpolation, and $\f_j$ and $\f_{j+1}$ are the two
% points of the frequency grid surrounding the image frequency,
% $\f_i'$. The row of the $\Hm$ matrix corresponding to $\f^i$ is then
% \begin{eqnarray}
%    \label{eq:sensor:mixer:hi}
%    \mat{h}^i &=& \frac{f_s(\f_i)}{f_s(\f_i)+f_s(\f_i')}  \\
%    \mat{h}^j &=& \frac{f_s(\f_i')}{f_s(\f_i)+f_s(\f_i')}
%                  \frac{\f_{j+1}-\f_i'}{\f_{j+1}-\f_j}     \nonumber \\
%    \mat{h}^{j+1} &=& \frac{f_s(\f_i')}{f_s(\f_i)+f_s(\f_i')}
%                  \frac{\f_i'-\f_j}{\f_{j+1}-\f_j}     \nonumber
% \end{eqnarray}
% where $\mat{h}^i$ is the value of $\mat{h}$ for frequency $\f_i$ etc.
% Remaining values of $\Hm$ are zero.

% For the special case when the image frequency matches perfectly a frequency
% grid point, the equations above can be simplified to give
% \begin{eqnarray}
%    \mat{h}^i &=& \frac{f_s(\f_i)}{f_s(\f_i)+f_s(\f_i')}    \nonumber \\
%    \mat{h}^j &=& \frac{f_s(\f_i')}{f_s(\f_i)+f_s(\f_i')}    \nonumber
% \end{eqnarray}


%\levelc{Practical solution} 
% \label{sec:sensor:mixer:practical}
 
% The responses of the sideband filter is determined by linear or cubic
% interpolation, dependent on the selected order.
% As the frequency in the primary band always equals one of the points
% of the monochromatic frequency grid, Equation
% \ref{eq:sensor:mixer:hi} can be used throughout. The weights for the
% image band are found by evaluating the polynomial basis from Equation
% \ref{eq:sensor:pbasis} at $\f_i'$ and multiplicate with 
% $f_s(\f_i') / (f_s(\f_i)+f_s(\f_i'))$. These calculations are 
% performed in the AMI function \verb|h_matrix|.

 
%\levelb{Brightness temperature} 
% \label{sec:sensor:tb}

% Some kind of calibration process, either in absolute or relative units,
% is always needed. For mm and sub-mm receivers, the calibration normally
% presents the measured intensity in some temperature scale, and conversion
% to brightness and Rayleigh-Jeans temperatures is also treated in this section.


%\levelc{Conversion to Planck brightness temperature} 
% \label{sec:sensor:tb_planck}

% The brightness temperature is defined as the temperature a blackbody 
% shall have to give the same intensity magnitude as observed. The 
% brightness temperature is thus calculated as
% \begin{equation}
%   T_b = \frac{h\f}{k_B} \frac{1}{\ln{ \left( \frac{2h\f^3}{c^2\mpbi}+1 \right)}}
%   \label{eq:sensor:cal:tb}
% \end{equation}
% where \mpbi\ is the radiative intensity.
 
% It should be noted that the conversion from intensity to brightness
% temperature is non-linear. This non-linearity has (at least) two important
% consequences:
% \begin{itemize}
%  \item The conversion from intensity to brightness temperature cannot be
%        included in \Hm.
%  \item \bf{Brightness temperature cannot be used for retrievals.}
% \end{itemize}
% Accordingly, the main reason to convert a spectrum to brightness
% temperatures is to display the spectrum in an unit that gives a more
% intuitive understanding of the emission magnitude.


%\levelc{Conversion to Rayleigh-Jean temperature} 
% \label{sec:sensor:tb:rj}

% For lower frequencies where $h\f \ll k_BT$ the Planck function can
% be approximated by the Rayleigh-Jean (RJ) formula:
% \begin{equation}
%   B \approx \frac{2\f^2k_BT}{c^2}
% \end{equation}
% This relationship holds rather well in the microwave region. For example,
% for $T=50$~K, $h\f = k_BT$ at 1.04~THz. The RJ approximation of the Planck
% function gives a natural definition on a ``brightness temperature'' with
% that has a linear relationship to the intensity:
% \begin{equation}
%   T_{rj} = \frac{c^2}{2\f^2k_B} \mpbi
%   \label{eq:sensor:cal:rj}
% \end{equation}
% This intensity unit is often referred to as the brightness temperature but
% to avoid confusion it is here denoted as the RJ temperature.
 
% As the intensity from intensity to RJ temperature is linear, this
% conversion can be included in \Hm\ and weighting functions can be
% converted using \ref{eq:sensor:cal:rj}, that is, retrievals are
% possible using RJ temperatures.
% On the other hand, the RJ temperature shall not be mistaken for the
% ``physical'' brightness temperature $(T_b)$ as the deviation between
% $T_b$ and $T_{rj}$ is not negligible \citep{eriksson:97a}.


%\levelb{Control file examples}
% \label{sec:sensor:cfe}

% The following sequence of ARTS functions can be used to store the
% spectra in both brightness temperature units:

% {\footnotesize
% \begin{verbatim}
%VectorCopy( y0, y ) {
%}
%yTRJ{
%}
%VectorWriteAscii( y ) {
%   "ytb_rj.aa"
%}
%VectorCopy( y, y0 ) {
%}
%yTB{
%}
%VectorWriteAscii( y ) {
%   "ytb_planck.aa"
%}
% \end{verbatim}
% }
% \noindent
% A weighting function matrix is converted to Rayleigh-Jean temperature
% as:

% {\footnotesize
% \begin{verbatim}
%MatrixTRJ( kx, kx ) {
%}

% \end{verbatim}
% }
 


%%% Local Variables: 
%%% mode: latex 
%%% TeX-master: "uguide" 
%%% End:

