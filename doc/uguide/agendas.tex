%
% To start the document, use
%  \levela{...}
% For lover level, sections use
%  \levelb{...}
%  \levelc{...}
%
\levela{WSVs, WSMs, and agendas}
 \label{sec:agendas}

%
% Document history, format:
%  \starthistory
%    date1 & text .... \\
%    date2 & text .... \\
%    ....
%  \stophistory
%
\starthistory
  020605 & Created by Stefan Buehler.\\
\stophistory


%
% Symbol table, format:
%  \startsymbols
%    ... & \artsstyle{...} & text ... \\
%    ... & \artsstyle{...} & text ... \\
%    ....
%  \stopsymbols
%
%
%\startsymbols
%  \Ind           & -                 & vector/matrix/tensor index           \\
%  \aInd{\Lat}    & -                 & the \Ind:th latitude                 \\
%  \VctLng        & -                 & vector length or size of matrix/tensor for a dimension \\
%  \aVctLng{\Lat} & -                 & length of the latitude grid \\
%  \Prs           & \artsstyle{p}     & pressure                             \\
%  \PrsAlt        & \artsstyle{pz}    & pressure altitude                    \\
%  \Rds           & \artsstyle{r}     & radius from the geoid centre         \\
%  \Alt           & \artsstyle{z}     & geometrical altitude above the geoid \\
%  \Lat           & \artsstyle{alpha} & latitude                             \\
%  \Lon           & \artsstyle{beta}  & longitude                            \\
%  \ZntAng        & \artsstyle{psi}   & zenith angle                         \\
%  \AzmAng        & \artsstyle{omega} & azimuth angle                      \\
% \label{symtable:ppath_defs}     
%\stopsymbols

This chapter deals with the main components of ARTS: \emph{Workspace
  variables}\index{workspace variables} (\textindex{WSVs}) and
\emph{workspace methods}\index{workspace methods} (\textindex{WSMs}).
Furthermore, it explains the use of \textindex{agendas}, a special
group of WSVs.



A propagation path is the way the radiation travels to reach the
sensor for a specified line-of-sight. A general description of
propagation paths is given in Section~\ref{sec:fm_defs:ppaths} and it
can be a good idea to read that section before continuing here. This
section describes how propagation paths are described and calculated.
In addition, at the end of the section some geodetic issues are
discussed, such as the choice of reference ellipsoid for the geoid.


\levelb{Implementation files}
%===
Variables and functions related to propagation paths are defined in the files:
\begin{itemize}
\item \fileindex{ppath.h}
\item \fileindex{ppath.cc}
\item \fileindex{m\_ppath.cc}
\item \fileindex{m\_atmosphere.cc}
\end{itemize}
The first file, \artsstyle{ppath.h}, contains the definition of the
structure to describe propagation paths, \typeindex{Ppath}. The
second file, \artsstyle{ppath.cc}, contains functions to perform
calculations to determine propagation paths. The third file,
\artsstyle{m\_ppath.cc}, contains the workspace functions related to
propagation path, but these functions mainly checks the input and the
actual calculations are performed by sub-functions in
\artsstyle{ppath.cc}. The fourth file, \artsstyle{m\_atmosphere.cc},
contains functions to set the geoid radius.




\levelb{Calculation approach}
%===================
\label{sec:ppath:approach}

A full propagation path is stored in the workspace variable
\wsvindex{ppath}, that is of the type \artsstyle{Ppath} (see next
section). The paths are determined by calculating a number of path
steps. A path step is the path from a point to the next crossing
of either the pressure, latitude or longitude grid. There is one
exception to this definition of a path step, and that is when there
is an intersection with a blackbody ground, which ends the propagation
path at that point. The starting point for the calculation of a path
step is normally a grid crossing point, but can also be an arbitrary
point inside the atmosphere, such as the sensor position. Only points
inside the model atmosphere are handled. The path steps are stored in
the workspace variable \wsvindex{ppath\_step}, that is of the same
type as \artsstyle{ppath}. The path steps are calculated by an agenda
called \wsvindex{ppath\_step\_agenda}. Example on functions that can
be used in \artsstyle{ppath\_step\_agenda} are
\wsmindex{ppath\_stepGeometric} and
\wsmindex{ppath\_stepGeometricWithLmax}.

Propagation paths are calculated with the workspace function
\wsmindex{ppathCalc}. The communication between this function and
\artsstyle{ppath\_step\_agenda} is handled by \artsstyle{ppath\_step}.
That variable is used both as input and output to
\artsstyle{ppath\_step\_agenda}.  The agenda gets back
\artsstyle{ppath\_step} as returned to \wsmindex{ppathCalc} and the
last path point hold by the structure is accordingly the starting
point for the new calculations. If a total propagation path shall be
determined, the agenda is called repeatedly until the starting point
of the propagation path is found and \artsstyle{ppath\_step} will hold
all path steps that together make up \artsstyle{ppath}. The starting
point is included in the returned structure.

The path is determined by starting at the end point and moving
backwards to the starting point. The calculations are initiated by
filling \artsstyle{ppath\_step} with the practical end point of the
path. This is either the position of the sensor (true or
hypothetical), or some point at the top of the atmosphere (determined
by geometrical calculations starting at the sensor). This
initialisation is not handled by \artsstyle{ppath\_step\_agenda}. 
The field \artsstyle{constant} is set by \artsstyle{ppathCalc}
to the correct value if the sensor is above the model atmosphere.
Otherwise, the field is set to be negative and is corrected by
\artsstyle{ppath\_step\_agenda} at the first call. This procedure is
needed as the propagation path constant changes if refraction is
considered, or not, when the sensor is placed inside the atmosphere.

The agenda performs only calculations to next crossing of a grid, all
other tasks are performed by \wsmindex{ppathCalc}, with one exception.
If there is an intersection of a blackbody ground, the calculations
stop at this point. This is flagged by setting the background field of
\artsstyle{ppath\_step}. Beside this, \wsmindex{ppathCalc} checks
if the starting point of the calculations is inside the scattering box
or below the ground level, and check if the last point of the path has
been reached. The starting point (the end furthest away from the
sensor) of a full propagation path can be the top of the atmosphere, a
blackbody ground and the cloud box.

The \artsstyle{ppath\_step\_agenda} put in points along the
propagation path at all crossings with the grids, tangent points and
points of ground reflection. There exists also functions for
\artsstyle{ppath\_step\_agenda}, such as
\artsstyle{ppath\_stepGeometricWithLmax}, that put in additional
points to ensure that the length along the path between the points
does not exceed a specified limit.



\levelb{The propagation path data structure}
%===================
\label{sec:ppath:Ppath}

A propagation path is represented by a structure of type
\typeindex{Ppath}. This structure holds also auxiliary variables to
facilitate the radiative transfer calculations and to speed up the
interpolation (if performed). The fields of \artsstyle{Ppath}, where
the data type is given inside square brackets, are:
\begin{description}

  \item[dim] [Index] The atmospheric dimensionality. This field shall always 
     be equal to the workspace variable \artsstyle{atmosphere\_dim}.
     
   \item[np] [Index] Number of positions to define the propagation
     path. The number of rows of \artsstyle{pos} and \artsstyle{los},
     and the length of \artsstyle{z}, \artsstyle{gp\_p},
     \artsstyle{gp\_lat} and \artsstyle{gp\_lon}, shall be equal to
     \artsstyle{np}. The length of \artsstyle{l\_step} is
     \artsstyle{np} - 1. If \artsstyle{np} $\leq$ 1, the observed
     spectrum is identical to the radiative background. For cases
     where the sensor is placed inside the model atmosphere and

     \artsstyle{np} = 1, the stored position is identical to the sensor
     position and that position can be used to determinate the radiative
     background (see below).

   \item[refraction] [Index] A flag (0 or 1) to indicate if refraction
     has been considered when determining the path. A value of 1 means
     that refraction has been considered.

   \item[method] [String] A string describing the calculation approach.
     For example, '1D basic geometrical'.
     
   \item[constant] [Numeric] The propagation path constant. Such a
     constant can be assigned to all geometrical paths and for 1D
     cases (with or without refraction). See
     Sections~\ref{sec:ppath:basicgeom} and [**]. This field can be
     initiated to a negative value to indicate that the constant is
     undefined or not yet set. For cases where the constant applies,
     \wsvindex{ppath\_step\_agenda} sets this constant at the first
     call of the agenda if the given value is negative.

   \item[pos] [Matrix] The position of the propagation path points.
     This matrix has \artsstyle{np} rows and up to 3 columns. Each row
     holds a position where column 1 is the radius, column 2 the
     latitude and column 3 the longitude (cf.
     Section~\ref{sec:fm_defs:sensorpos}). The number of columns for
     1D and 2D is 2, while for 3D it is 3. The latitudes are stored
     for 1D cases as these can be of interest for some applications
     and are useful if the propagation path shall be plotted. The
     latitudes for 1D give the angular distance to the sensor (see
     further Section~\ref{sec:fm_defs:atmdim}).
     
     The propagation path is stored in reversed order, that is, the
     position with index 0 is the path point closest to the sensor
     (and equals the sensor position if it is inside the atmosphere).
     The full path is stored also for 1D cases with symmetry around a
     tangent point (in contrast to ARTS-1). 
     
  \item[z] [Vector] The geometrical altitude for each path position. The
     length of this vector is accordingly \artsstyle{np}. This is a help
     variable for plotting and similar purposes. It shall not be used to
     interpolate the atmospheric fields, as pressure is the main altitude
     coordinate.

  \item[l\_step] [Vector] The length along the propagation path between
     the positions in \artsstyle{pos}. The length of the vector is
     \artsstyle{np} - 1. 
     
   \item[gp\_p] [ArrayOfGridPos] Index position with respect to the
     pressure grid. The structure for grid positions is described in
     Section~\ref{sec:interpolation:gridpos}. 
     
   \item[gp\_lat] [ArrayOfGridPos] As \artsstyle{gp\_p} but with
     respect to the latitude grid.

   \item[gp\_lon] [ArrayOfGridPos] As \artsstyle{gp\_p} but with
     respect to the longitude grid.
     
   \item[los] [Matrix] The line-of-sight of the propagation path at
     each point. The number of rows of the matrix is \artsstyle{np}.
     For 1D and 2D, the matrix has a single column holding the zenith
     angle. For 3D there is an additional column giving the azimuth
     angle. The zenith and azimuth angles are defined in
     Section~\ref{sec:fm_defs:los}. If the radiative background is the
     cloud box, the last position (in \artsstyle{pos}) and
     line-of-sight give the relevant information needed when
     extracting the radiative background from the cloud box intensity
     field.
     
   \item[background] [String] The radiative background for the
     propagation path. The possible
     options for this field are 'space', 'blackbody ground', 'cloud
     box interior' and 'cloud box surface', where the source of
     radiation should be clear the content of the strings.

  \item[ground] [Index] A boolean to indicate that there is an intersection
     of the path by the ground. If the ground is treated to be a blackbody,
     \artsstyle{ground} can only have the value of 0.
     
   \item[i\_ground] [Index] The position index of the ground.
     Undefined if \artsstyle{ground} = 0. Row \artsstyle{i\_ground} of
     \artsstyle{pos} gives the position of the ground reflection, which
     is needed for determining the ground emissivity for 2D and 3D cases.
     
   \item[tan\_pos] [Vector] The position of the tangent point. This
     vector is only set if there exists a tangent point (above the
     ground level), the length of the vector is otherwise 0.
     
   \item[geom\_tan\_pos] [Vector] The position of the geometrical
     tangent point. This vector is set for all downward observations.
     Refraction and ground reflections are neglected when calculating
     this tangent point position. This field is not covered by
     \artsstyle{ppath\_step\_agenda}.
     
    \item[symmetry] [Index] A boolean to indicate that the atmospheric
      conditions are symmetric around some point. This will the case
      for 1D and downward observations. This boolean shows only that a
      symmetry point exists, it will not affect the later
      calculations.  To make use of the symmetry on the radiative
      transfer calculations, the workspace variable [* to be defined
      *] must be set to 1.

   \item[i\_symmetry] [Index] The index of the symmetry point if 
     \artsstyle{symmetry}=1. Otherwise undefined.

\end{description}




\levelb{Some basic geometrical relationships}
%===================
\label{sec:ppath:basicgeom}

This section gives some expressions to determine positions along a
propagation path when refraction is neglected. The expressions deal
only with propagation path inside a plane, where the latitude angle is
the angular distance from an arbitrary point. This means that the
expressions given here can be directly applied for 1D and 2D, but not for
3D. The ARTS function for making the calculation of concern is given
inside parenthesis above each equation, if not stated explicitly. A
part of a geometrical propagation path is shown in
Figure~\ref{fig:ppath:1d2dgeom}.

\begin{figure}[!t]
 \begin{center}
  \begin{minipage}[c]{0.65\textwidth}
   \begin{center}
    \includegraphics*[width=0.9\hsize]{Figs/ppath/geom1d}
   \end{center}
  \end{minipage}%
  \begin{minipage}[c]{0.35\textwidth}
   \caption{The radius (\Rds) and zenith angle (\ZntAng) for two points along
     the propagation path, and the distance along the path ($\Delta\PpathLng$)
     and the latitude difference ($\Delta\Lat$) between these points.}
   \label{fig:ppath:1d2dgeom}
  \end{minipage}
 \end{center}
\end{figure}   

The law of sines gives that the product must $\Rds\sin(\ZntAng)$ be
constant along the propagation path:
\begin{equation}
  p_c = \Rds\sin(\ZntAng)
  \label{eq:ppath:geomconst}
\end{equation}
where the absolute value is taken for 2D zenith angles as they can for
such cases be negative. The propagation path constant, $p_c$, is
determined by the position and line-of-sight of the sensor, a
calculation done by the function \funcindex{geometrical\_ppc}. The
constant equals also the radius of the tangent point of the path (that
is found along an imaginary prolongation of the path behind the sensor
if the viewing direction is upwards). The expressions below are based
on $p_c$ as the usage of a global constant for the path should
decrease the sensitivity to numerical inaccuracies. If the
calculations are based solely on the values for the neighbouring
point, a numerical inaccuracy can accumulate when going from one point
to next. The propagation path constant is stored in the field
\artsstyle{constant} of \wsvindex{ppath} and \wsvindex{ppath\_step}.

The relationship between the distance along the path for an
infinitesimal change in radius is here denoted as the
\textindex{geometrical factor}, $g$. If refraction is neglected, valid
expressions for the geometrical factor are
\begin{equation}
  g = \frac{\DiffD l}{\DiffD r} 
           = \frac{1}{\cos(\ZntAng)} = \frac{1}{\sqrt{1-\sin^2(\ZntAng)}}
                                            = \frac{\Rds}{\sqrt{\Rds^2-p_c^2}}
  \label{eq:ppath:g_geom}
\end{equation}
For the radiative transfer calculations, only the distance between the
points, $\Delta \PpathLng$, is of interest, but for the internal
propagation path calculations the length from the tangent point (real
or imaginary), \PpathLng, is used. By integrating
Equation~\ref{eq:ppath:g_geom}, we get that
(\funcindex{geomppath\_l\_at\_r})
\begin{equation}
  \PpathLng(\Rds) = \sqrt{\Rds^2-p_c^2} 
  \label{eq:ppath:r2l}
\end{equation}
As refraction is here neglected, the tangent point, the point of
concern and the centre of the coordinate system make up a right
triangle and Equation~\ref{eq:ppath:r2l} corresponds to the
Pythagorean relation where $p_c$ is the radius of the tangent point.
The distance between two points ($\Delta \PpathLng$) is obtained by
taking the difference of Equation~\ref{eq:ppath:r2l} for the two
radii.

The radius for a given \PpathLng\ is simply (\funcindex{geomppath\_r\_at\_l})
\begin{equation}
  \Rds(\PpathLng) = \sqrt{\PpathLng^2+p_c^2} 
  \label{eq:ppath:l2r}
\end{equation}
The zenith angle for a given radius is (\funcindex{geomppath\_za\_at\_r})
\begin{equation}
  \ZntAng(\Rds) = \left\{
   \begin{array}{ll}
    180 - \sin^{-1}(p_c/\Rds) & 
                   \textrm{for}\quad 90\degree < \aZntAng{a} \leq 180\degree\\
    \sin^{-1}(p_c/\Rds) & 
                   \textrm{for}\quad 0\degree \leq \aZntAng{a} \leq 90\degree\\
    -\sin^{-1}(p_c/\Rds) & 
                   \textrm{for}\quad -90\degree \leq \aZntAng{a} < 0\degree\\
    \sin^{-1}(p_c/\Rds) - 180 & 
                  \textrm{for}\quad -180\degree \leq \aZntAng{a} < -90\degree\\
   \end{array}   \right.
  \label{eq:ppath:r2psi}
\end{equation}
where \aZntAng{a} is any zenith angle valid for the path on the same
side of the tangent point. For example, for a 1D case, the part of the
path between the tangent point and the sensor has zenith angles
$90\degree < \aZntAng{a} \leq 180\degree$.

The latitude for a point is most easily determined by its zenith angle
\funcindex{geomppath\_lat\_at\_za})
\begin{equation}
  \Lat(\ZntAng) = \aLat{0} + \ZntAng - \aZntAng{0} 
  \label{eq:ppath:za2lat}
\end{equation}
where \aZntAng{0} and \aLat{0} are the zenith angle and latitude of some 
other point of the path. Equation~\ref{eq:ppath:za2lat} is based on the 
fact that the quantities \aZntAng{1}, \aZntAng{2} and $\Delta\Lat$
fulfil the relationship
\begin{equation}
  \Delta\Lat = \aZntAng{1} - \aZntAng{2},
\end{equation}
this independently of the sign of the zenith angles. The definitions
used here result in that the absolute value of the zenith angle always
decreases towards zero when following the path in the line-of-sight
direction, that is, when going away from the sensor. It should then be
remembered that the latitudes for 1D measures the angular distance to
the sensor, and for 2D a positive zenith angle means observation
towards higher latitudes.




\levelb{Calculation of propagations path steps}
%===================
\label{sec:ppath:stepcalc}

This section describes the calculation of the propagation path steps
for different atmospheric dimensionalities, and when refraction is
considered or not. The case when refraction is neglected is also
denoted as geometrical calculations. The workspace functions for
propagation path step calculations adjust automatically to the
atmospheric dimensionality, but the actual calculations are performed
a sub-functions for each dimensionality.

Workspace functions to calculate geometrical propagation steps are:
\begin{itemize}
\item \artsstyle{ppath\_stepGeometric}
\item \artsstyle{ppath\_stepGeometricWithLmax}
\end{itemize}


\levelc{Some general remarks}
%===================

The grid positions are calculated on the same time as the path is
determined. The main reason to this is that the grid positions make it
possible to quickly determine inside which grid box the path step is
found. Without the grid positions, each call of the functions would
need a costly search to locate the starting position in the grids.  If
you are not familiar with grid positions, it is recommended to read
Section~\ref{sec:interpolation} before you continue here.

The limited numerical accuracy requires some care when setting the
grid positions. First of all, rounding errors can give a fractional
distance $< 0$ or $> 1$ and this must be avoided. The function
\funcindex{gridpos\_check\_fd} was created for this purpose, and
should be called for each grid position. This function just
sets all values below 0 to 0 and all value above 1 to 1. In addition,
the grid position for the end point of a path step (beside when there
is an intersection with a blackbody ground) must have a fractional
distance of exactly 0 or 1, but this is not ensured by
\artsstyle{gridpos\_check\_fd} and for end points the function
\funcindex{gridpos\_force\_end\_fd} shall also be called. There will
be an error if \artsstyle{gridpos\_force\_end\_fd} is called with
original fractional distances deviating more than 1\topowerten{-6}
from 0 and 1.

Some care is needed to determine in which grid range a path step is
found. First of all, there exists an ambiguity for the fractional
distance at the grid points. It can either be 0 or 1. In addition, if
a position is exactly on top of a grid point, the observation
direction determines the interesting grid range. As an help to resolve
these question there is the function \funcindex{gridpos2gridrange}.
This function takes an argument describing the direction of the
line-of-sight with respect to the grids. This argument shall be set to
1 if the viewing direction is towards higher indexes. The direction
argument can be set with the following logical expressions, for the
different combinations of atmospheric dimensionality and grid of
interest:

 {\bf 1D-3D, pressure}: $\quad |\ZntAng| \leq 90\degree$

 {\bf 2D, latitude}: $\quad \ZntAng \geq 0\degree$

 {\bf 3D, latitude}: $\quad \AzmAng \leq 90\degree$

 {\bf 3D, longitude}: $\quad \AzmAng \geq 0\degree$


\levelc{1D without refraction}
%===================
\label{sec:ppath:1Dwithout}

The calculations are outlined in Algorithm~\ref{alg:ppath:1Dgeom}. The
basic problem is to determine the end and turning radii for the
propagation path. When these radii are determined, it is
straightforward to determine the quantities to be included in the
\artsstyle{Ppath} structure. Needed mathematics are given by
Equations~\ref{eq:ppath:geomconst}~-~\ref{eq:ppath:za2lat}.

\begin{algorithm}[!t]
 \begin{algorithmic}
  \STATE{set \aRds{start} to match the last position of input \artsstyle{ppath} structure}
  \IF{\artsstyle{ppath.constant} $< 0$}
   \STATE{calculate the propagation path constant}
  \ENDIF
  \STATE{determine the lower index of the interesting pressure grid range, \aInd{0}}
  \STATE{create a vector holding a minimum set of radii to describe the path, \aRds{s}:}
  \IF{$\aZntAng{start} \leq 90\degree$}
   \STATE{$\aRds{s} = \{\,\aRds{start}\,,\,\Rds[\aInd{0}+1]\,\}$}
   \COMMENT{$\Rds[\aInd{0}+1]$ = radius for pressure level with index $\aInd{0}+1$}
  \ELSE
   \STATE{calculate radius for tangent point, \aRds{tan}}
   \STATE{calculate radius for ground, \aRds{ground}}

   \IF{ \aRds{tan} $\geq$ both \Rds[\aInd{0}] and \aRds{ground}}
    \STATE{$\aRds{s} = \{\,\aRds{start}\,,\,\aRds{tan}\,,\,\Rds[\aInd{0}+1]\,\}$}
   \ELSE
    \IF{ \Rds[\aInd{0}] $>$ \aRds{ground} }
     \STATE{$\aRds{s} = \{\,\aRds{start}\,,\,\Rds[\aInd{0}]\,\}$}
    \ELSE
     \IF{ blackbody ground }
      \STATE{$\aRds{s} = \{\,\aRds{start}\,,\,\aRds{ground}\,\}$}
     \ELSE
      \STATE{$\aRds{s} = \{\,\aRds{start}\,,\,\aRds{ground}\,,\,\Rds[\aInd{0}+1]\,\}$}
     \ENDIF
    \ENDIF
   \ENDIF
  \ENDIF
  \STATE{calculate the distance to the tangent point for the radii in \aRds{s}}
  \IF{length criterion set}
   \STATE{calculate needed number of sub-steps between the radii in \aRds{s}}
  \ELSE
   \STATE{the number of needed sub-steps is throughout 1}
  \ENDIF
  \FORALL{radii in \aRds{s}}
   \FORALL{sub-steps}
    \STATE{take radii from \aRds{s} or calculate by Eq. \ref{eq:ppath:l2r}}
    \STATE{use Eqs. \ref{eq:ppath:r2psi} and \ref{eq:ppath:za2lat} to determine zenith angle and latitude}
    \STATE{calculate vertical grid position}
   \ENDFOR
  \ENDFOR
 \end{algorithmic}
 \caption{Outline of the calculation of 1D geometrical propagation path steps.}
 \label{alg:ppath:1Dgeom}
\end{algorithm}



\levelc{1D with refraction}
%===================
\label{sec:ppath:1Dwith}

Not yet implemented.


\levelc{2D without refraction}
%===================
\label{sec:ppath:2Dwithout}

Not yet implemented.


\levelc{2D with refraction}
%===================
\label{sec:ppath:2Dwith}

Not yet implemented.


\levelc{3D without refraction}
%===================
\label{sec:ppath:3Dwithout}

Not yet implemented.


\levelc{3D with refraction}
%===================
\label{sec:ppath:3Dwith}

Not yet implemented.


\levelb{Geoid ellipsoids and geodetic datums}
%===================
\label{sec:ppath:geoids}

This section defines the geoid ellipsoid and discusses related
issues. The geoid is introduced in Section~\ref{sec:fm_defs:geoid}.
The workspace variable representing the geoid is \wsvindex{r\_geoid}.


\levelc{Geoid ellipsoids}
%===================
\label{sec:ppath:geoid}

All geodetic datums are based on a reference ellipsoid\index{geoid
  ellipsoid}. The ellipsoid is rotationally symmetric around the
north-south axis. That is, the ellipsoid radius has no longitude
variation, it is only a function of latitude. The ellipsoid is
described by an equatorial radius, \aRds{e}, and a polar radius,
\aRds{p}. These radii are indicated in Figure~\ref{fig:ppath:lats}.
The radius of the ellipsoid for a given latitude is
\begin{equation}
 \aRds{\odot}(\Lat) = \sqrt{\frac{\aRds{e}^2\aRds{p}^2}
                    {\aRds{e}^2\sin^2\Lat+\aRds{p}^2\cos^2\Lat}}
 \label{eq:ppath:ellipsradius} 
\end{equation}
The radius given by Equation~\ref{eq:ppath:ellipsradius} can be
directly applied for 2D and 3D cases. On the other hand, for 1D cases
the reference geoid is by definition a sphere and the radius of this
sphere shall be selected in such way that it represents the local
shape of a reference ellipsoid. This is achieved by setting
\aRds{\odot} to the radius of curvature of the ellipsoid. The
curvature radius differs from the local radius except at the equator
and an east-west direction. For example, at the equator and a
north-south direction, the curvature radius is smaller then the local
radius, while at the poles (for all directions) it is greater
(see further Figure~\ref{fig:ppath:wgs84radii}). 

The \textindex{curvature radius}, \aRds{c}, of an ellipsoid is 
\citep{rodgers:00}
\begin{equation}
 \aRds{c} = \frac{1}{\aRds{ns}^{-1}\cos^2 \Lat + \aRds{ew}^{-1}\sin^2 \Lat}
 \label{eq:ppath:curvradius} 
\end{equation}
where \aRds{ns} and \aRds{ew} are the north-south and east-west curvature radius, respectively,
\begin{eqnarray}
 \aRds{ns} &=& \aRds{e}^2\aRds{p}^2 (
           \aRds{e}^2\cos^2\AzmAng+\aRds{p}^2\sin^2\AzmAng )^{-\frac{3}{2}} \\
 \aRds{ew} &=& \aRds{e}^2 (
           \aRds{e}^2\cos^2\AzmAng+\aRds{p}^2\sin^2\AzmAng )^{-\frac{1}{2}} 
 \label{eq:ppath:rew} 
\end{eqnarray}
The azimuth angle, \AzmAng, is defined in
Section~\ref{sec:fm_defs:los}. The latitude and azimuth angle to
apply in Equations \ref{eq:ppath:curvradius}~-~\ref{eq:ppath:rew}
shall rather be valid for a middle point of the propagation paths
(such as some tangent point), instead of the sensor position. 

\begin{figure}[!p]
 \begin{center}
  \begin{minipage}[c]{0.65\textwidth}
   \begin{center}
    \includegraphics*[width=0.9\hsize]{Figs/ppath/latitudes}
   \end{center}
  \end{minipage}%
  \begin{minipage}[c]{0.35\textwidth}
   \caption{Definition of the ellipsoid radii, \aRds{e} and \aRds{p}, 
     geocentric latitude, \Lat, and geodetic latitude, \Lat$^*$. The
     dotted line is the normal to the local tangent of the geoid
     ellipsoid. The zenith and nadir directions, and geometrical
     altitudes, are here defined to follow the solid line.}
   \label{fig:ppath:lats}
  \end{minipage}
 \end{center}
\end{figure}   

\begin{figure}[!p]
 \begin{minipage}[c]{0.65\textwidth}
 \includegraphics*[width=0.96\textwidth]{ppath/wgs84_radii}
 \end{minipage}%
 \begin{minipage}[c]{0.35\textwidth}
  \caption{The ellipsoid radius (\aRds{\odot}) and curvature radius (\aRds{c})
    for the
    WGS-84 reference ellipsoid. The curvature radii are valid for the
    north-south direction.}
  \label{fig:ppath:wgs84radii}
 \end{minipage}%
\end{figure}   
        
\begin{figure}[!p]
 \begin{minipage}[c]{0.65\textwidth}
 \includegraphics*[width=0.96\textwidth]{ppath/wgs84_latdiff}
 \end{minipage}%
 \begin{minipage}[c]{0.35\textwidth}
  \caption{The change of the WGS-84 ellipsoid radius for  1\degree\ 
            latitude differences.}
  \label{fig:ppath:latdiff}
 \end{minipage}%
\end{figure}   



\levelc{Geocentric and geodetic latitudes}
%===================
\label{sec:ppath:geolat}

The fact that the geoid is an ellipsoid, instead of a sphere, opens up
for the two different definitions of the latitude. The
\textindex{geocentric latitude}, which is the the one used here, is the
angle between the equatorial plane and the vector from the coordinate
system centre to the position of concern. The \textindex{geodetic
  latitude} is also defined with respect to the equatorial plane, but
the angle to the normal to the reference ellipsoid is considered here, as
shown in Figure~\ref{fig:ppath:lats}. It could be mentioned that a
geocentric latitude does not depend on the geoid ellipsoid used, while
the geodetic latitudes change if another reference ellipsoid is
selected. An approximative relationship between the geodetic
($\Lat^*$) and geocentric (\Lat) latitudes is \citep{montenbruck:00}
\begin{equation}
 \Lat^* = \Lat + f\,\sin(2\Lat)  
 \label{eq:ppath:lats}
\end{equation}
where $f$ is the flattening of the ellipse:
\begin{equation}
 f = \frac{\aRds{e}-\aRds{p}}{\aRds{e}}
 \label{eq:ppath:flattening}
\end{equation}
The value of $f$ for the Earth is about 1/298.26. This means that the
largest differences between \Lat\ and $\Lat^*$ are found at
mid-latitudes and the maximum value is about 12 arc-minutes.

The \textindex{zenith} and \textindex{nadir} directions shall normally be
defined to follow the normal to the reference ellipsoid, but, if
nothing else is mentioned, these directions are here treated to go
along the vector the centre of the coordinate system, as indicated in
Figure~\ref{fig:ppath:lats}. This latter definition is preferred
as it results in that a propagation path in the zenith/nadir direction
can be described by a single latitude and longitude value. The
difference in geometrical altitude when using these two possible
definitions on the zenith direction is proportional to the deviation
between geocentric and geodetic latitude (Equation~\ref{eq:ppath:lats}).
For an altitude of 100~km around $\Lat=45\degree$, the difference is
about 350~m.


\levelc{Geodetic datums}
%===================
\label{sec:ppath:geodatums}

Table~\ref{tab:ppath:geodatums} gives the equatorial and polar radii
of the reference ellipsoid for the geodetic datums handled by ARTS.

\begin{table}[!h]
  \begin{center}
    \begin{tabular}{c c c c l}
     Datum & \aRds{e} & \aRds{p} & $1/f$ & Reference \vspace*{1mm} \\ 
     \hline 
     WGS-84 & 6378.137 km & \emph{6356.752 km} & 298.2572235 & {\small \citet{montenbruck:00}}  \rule{0mm}{5mm} \vspace*{1mm} \\
     \hline
    \end{tabular}
    \caption{Equatorial and polar radius of reference ellipsoids. Values 
      given as \emph{italic} are 
      derived by the other two values and Equation~\ref{eq:ppath:flattening}.}
    \label{tab:ppath:geodatums}
  \end{center}
\end{table}



\levelb{Control file examples}
%===================
\label{sec:ppath:cfile}




%%% Local Variables: 
%%% mode: latex
%%% TeX-master: "uguide"
%%% End: 

