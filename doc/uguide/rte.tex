\chapter{Clear sky radiative transfer}
 \label{sec:rte}


\starthistory
  ? & Started by ?. \\
\stophistory


This chapter deals with the ``clear sky'' part of ARTS. That is, the
only physical features to consider are absorption and emission. A
simulation can consists solely of clear sky calculations. A clear sky
calculations can also treat the radiative transfer from the surface or
the cloud box, to the sensor. How these calculations shall be
performed is set by defining \artsstyle{rte\_agenda}. 

So far only unpolarised absorption (absorption does not depend on
polarisation state) is treated. Some steps towards handling Zeeman
splitting (a feature affecting oxygen absorption) have been taken, but
the work has not yet been finished.



\section{The vector radiative transfer equation}
%---
\label{sec:rte:vrte}

The complete vector radiative transfer, including scattering, is given
by Equation~\ref{eq:rtetheory:VRTE}. If scattering can be neglected,
the equation can be written as
\begin{equation}
  \label{eq:rte:vrte}
  \frac {\DiffD\StoVec}{\DiffD s} = -\ExtMat\StoVec + \AbsVec B,
\end{equation}
where \StoVec\ is the intensity vector (the Stokes vector), $s$ is the
distance along the propagation path, \ExtMat\ is the extinction
matrix, \AbsVec\ is the absorption vector and $B$ is the source
function (a scalar). If local thermodynamic equilibrium applies, $B$
equals the Planck function describing blackbody radiation. See further
Chapter~\ref{sec:rte_theory}.

As scattering here is neglected, the elements of \ExtMat\ and \AbsVec\ 
are linked to each other, and we have that\footnote{This equation is
  for sure valid for unpolarised absorption. We have not confirmed
  generally for polarised absorption, but appears to be valid for
  Zeeman splitting.}:
\begin{eqnarray*}
  \ExtMat^{-1}\AbsVec &=& \left[\begin{array}{c} \alpha\\0\\0\\0 \end{array}\right],
\end{eqnarray*}
where $\alpha$ is the total gas absorption coefficient. 



\section{Standard algorithm}
%---
\label{sec:rte:std}





\subsection{Simulation of transmission measurements}
%===================
\label{sec:rte:trans_sim}

\dots


%%% Local Variables: 
%%% mode: latex
%%% TeX-master: "main"
%%% End: 
