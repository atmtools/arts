\chapter{Clear sky radiative transfer}
 \label{sec:rte}


\starthistory
  ? & Started by ?. \\
\stophistory


This chapter deals with the ``clear sky'' part of ARTS. That is, the
only physical features to consider are absorption and emission. A
simulation can consists solely of clear sky calculations. A clear sky
calculations can also treat the radiative transfer from the surface or
the cloud box, to the sensor. How these calculations shall be
performed is set by defining \artsstyle{rte\_agenda}. 


\section{The vector radiative transfer equation}
%---
\label{sec:rte:vrte}

The complete vector radiative transfer, including scattering, is given
by Equation~\ref{eq:rtetheory:VRTE}. If scattering can be neglected,
the equation can be written as
\begin{equation}
  \label{eq:rte:vrte}
  \frac {\DiffD\StoVec}{\DiffD s} = -\ExtMat\StoVec + \AbsVec B,
\end{equation}
where \StoVec\ is the intensity vector (the Stokes vector), $s$ is the
distance along the propagation path, \ExtMat\ is the extinction
matrix, \AbsVec\ is the absorption vector and $B$ is the source
function (a scalar). If local thermodynamic equilibrium applies, $B$
equals the Planck function describing blackbody radiation. See further
Chapter~\ref{sec:rte_theory}.

As scattering here is neglected, the elements of \ExtMat\ and \AbsVec\ 
are linked to each other, and we have that:
\begin{eqnarray*}
  \ExtMat^{-1}\AbsVec &=& \left[\begin{array}{c} \alpha\\0\\0\\0 \end{array}\right],
\end{eqnarray*}
where $\alpha$ is the total gas absorption coefficient. 


\subsection{Simulation of transmission measurements}
%===================
\label{sec:rte:trans_sim}

\dots


%%% Local Variables: 
%%% mode: latex
%%% TeX-master: "main"
%%% End: 
