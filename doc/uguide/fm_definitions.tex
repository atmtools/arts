%
% To start the document, use
%  \levela{...}
% For lover level, sections use
%  \levelb{...}
%  \levelc{...}
%
\levela{Forward model concepts and definitions}
 \label{sec:fm_defs}

%
% Document history, format:
%  \starthistory
%    date1 & text .... \\
%    date2 & text .... \\
%    ....
%  \stophistory
%
\starthistory
  02xxxx & xxx.\\
\stophistory

This chapter introduces terms and concepts of ARTS as a forward model,
in contrast to the previous chapter that describes ARTS as a computer
programme. While the content of the previous chapter is specific for
ARTS, as the way to use a forward model programme differ normally
significantly from one implementation to another, this chapter is of
more general nature. Most of the quantities treated here should be
part of any forward model of the same complexity as ARTS, where only
details regarding the definition should differ. The aim of this chapter
is to describe important terms and concepts in such way that the 
content of this user guide can be fully appreciated and that you shall
understand how to generate a control file for your simulation problem. 




\levelb{Defining the atmosphere}
%===================
\label{sec:fm_defs:atmosphere}


 \begin{figure}
  \begin{center}
   \includegraphics*[width=0.95\hsize]{Figs/atm_dim_1d}
   \caption{ Schematic of a 1D atmosphere. The atmosphere is here circularly
     symmetric. This means that the radius of the geoid, the ground
     and all the pressure surfaces is constant. The cloud box extends
     from the ground up to the thin solid line. The upper limit of
     the cloud box coincides always with a pressure surface. }
   \label{fig:fm_defs:1d}  
  \end{center}
 \end{figure}
 % This figure was produced by the Matlab function mkfigs_atm_dims.

 \begin{figure}
  \begin{center}
   \includegraphics*[width=0.95\hsize]{Figs/atm_dim_2d}
   \caption{ Schematic of a 2D atmosphere. The radii of the geoid, the ground
     and all the pressure surfaces are for 2D atmospheres allowed to have a
     latitude variation. The limits of
     the cloud box coincide always with grid box boundaries. Plotting
     symbols as in Figure~\ref{fig:fm_defs:1d}. }
   \label{fig:fm_defs:2d}  
  \end{center}
 \end{figure}
 % This figure was produced by the Matlab function mkfigs_atm_dims.



%%% Local Variables: 
%%% mode: latex
%%% TeX-master: "uguide"
%%% End: 
