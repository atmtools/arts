%
% To start the document, use
%  \levela{...}
% For lover level, sections use
%  \levelb{...}
%  \levelc{...}
%
\levela{The forward model: concepts, definitions and overview}
 \label{sec:fm_defs}

%
% Document history, format:
%  \starthistory
%    date1 & text .... \\
%    date2 & text .... \\
%    ....
%  \stophistory
%
\starthistory
  020314 & Layout made by Patrick Eriksson.\\
\stophistory


%
% Symbol table, format:
%  \startsymbols
%    ... & \verb|...| & text ... \\
%    ... & \verb|...| & text ... \\
%    ....
%  \stopsymbols
%
%
\startsymbols
  \Ind           & -                 & vector/matrix/tensor index           \\
  \aInd{\Lat}    & -                 & the \Ind:th latitude                 \\
  \VctLng        & -                 & vector length or size of matrix/tensor for a dimension \\
  \aVctLng{\Lat} & -                 & length of the latitude grid \\
  $e_g$          & \verb|e_ground|   & ground emissivity                    \\
  \Prs           & \verb|p|          & pressure                             \\
  \PrsAlt        & \verb|pz|         & pressure altitude                    \\
  \Rds           & \verb|r|          & radius from the centre of the coordinate system         \\
  \Tmp           & \verb|t|          & temperature \\
  \Alt           & \verb|z|          & geometrical altitude above the geoid \\
  \Planck        & -                 & Planck function                      \\
  \Lat           & \verb|alpha|      & latitude                             \\
  \Lon           & \verb|beta|       & longitude                            \\
  \ZntAng        & \verb|psi|        & zenith angle                         \\
  \AzmAng        & \verb|omega|      & azimuthal angle                      \\
 \label{symtable:fm_defs}     
\stopsymbols



This chapter introduces terms and concepts of ARTS as a forward model,
in contrast to the previous chapter that describes ARTS as a computer
programme. While the content of the previous chapter is specific for
ARTS, as the way to use a forward model programme differ normally
significantly from one implementation to another, this chapter is of
more general nature. Most of the quantities treated here should be
part of any forward model of the same complexity as ARTS, where only
details regarding the definition should differ. The aim of this
chapter is to give an overview of the forward model and to describe
important terms and concepts, in such way that, the content of this user
guide can be fully appreciated and that you shall understand how to
construct a control file for your simulation problem.



\levelb{Mathematics}
%==============================================================================
\label{sec:fm_defs:math}

[* A short introduction to the vectors, matrices and tensors should be
found here. It should be described what the terms means, the order in
which dimensions are given, how sizes are specified here in the guide,
how different dimensionalities are handled etc. 

Vectors, matrices and tensors are indexed by round braces, e.g.
\begin{equation}
  \Alt(\aInd{\Prs},\aInd{\Lat},\aInd{\Lon})
\end{equation}
This is not consistent with how vectors are indexed in ARTS, but it
would be confusing to use square braces, this as square braces are
used for sizes. A size is given as $
[\aVctLng{\Prs},\aVctLng{\Lat},\aVctLng{\Lon}]$, or the size:
\begin{equation}
  \Alt \SzeSmb [\aVctLng{\Prs},\aVctLng{\Lat},\aVctLng{\Lon}]
\end{equation}
The concept of rows, columns, pages etc. should be described so a
particular dimension can be identified in a consistent manner.

The indexing is 0-based as in ARTS. *]



\levelb{The atmosphere}
%==============================================================================
\label{sec:fm_defs:atmosphere}


\levelc{Atmospheric dimensionality}
%===================
\label{sec:fm_defs:atmdim}

The structure of the atmosphere can be defined to have different
degree of complexity, the \qindex{atmospheric dimensionality}. There
exist three levels for the complexity of the atmosphere, 1D, 2D and
3D, where 1D and 2D can be treated as special cases of 3D. The
significance of these different atmospheric dimensionalities, and the
coordinate systems used, are described below in this section. The
atmospheric dimensionality is selected by setting the workspace
variable \verb|atmosphere_dim| to a value between 1 and 3. Variables
for which the size depends on the atmospheric dimensionality are
checked, when used, to have a size consistent with
\verb|atmosphere_dim|.

\begin{description}
  
\item[\qindex{3D}\,\,\,] In this, the most general, case, the
  atmospheric fields vary in all three spatial coordinates, as in a
  true atmosphere. A \qindex{spherical coordinate system} is used where the
  dimensions are radius (\Rds), latitude (\Lat) and longitude (\Lon),
  and a position is given as $(\Lon,\Lat,\Rds)$. With other words, the
  standard way to specify a geographical position is followed.
  However, the way to specify the radial position differs depending on
  the context, which is described in
  Section~\ref{sec:fm_defs:altitudes}. The valid range for latitudes
  is [-90\degree,+90\degree], where +90\degree corresponds to the
  North pole etc. Longitudes are counted from the Greenwich meridian
  with positive values towards the east. Longitudes can have values
  from -360\degree to +360\degree. When the difference between the
  last and first value of the longitude grid is $\geq 360\degree$ then
  the whole globe is considered to be covered. The user must ensure
  that the atmospheric fields for \Lon and $\Lon+360\degree$ are
  equal. If a point of propagation path is found to be outside the
  range of the longitude grid, this will results in an error if not
  the whole globe is covered. Otherwise, the longitude is shifted with
  360\degree in the relevant direction.

\item[\qindex{1D}\,\,\,] A 1D atmosphere can be described as being
  spherically symmetric. The term 1D is used here for simplicity and
  historical reasons, not because it is a true 1D case (a strictly 1D
  atmosphere would just extend along a line). A spherical symmetry
  means that atmospheric fields and the ground extend in all three
  dimensions, but they have no latitude and longitude variation. This
  means that, for example, atmospheric fields vary only as a function
  of altitude and the ground constitutes the surface of a sphere. The
  radial coordinate is accordingly sufficient when dealing with
  atmospheric quantities, but the angular distance between the sensor
  and a point along the propagation path can be of interest, for
  example when determining the cross-link between two satellites (a
  fact that shows that this is not a true 1D case). A polar coordinate
  system is for this reason used when describing propagation paths,
  where the coordinate additional to the radius gives the angular
  distance inside the viewing plane between the sensor and the point
  of interest (see further Section~[**]). This latter coordinate
  system coincides with the one used for 2D where the sensor position
  is set to be the zero point for latitude. A 1D atmosphere is shown
  in Figure~\ref{fig:fm_defs:1d}.
  
\item[\qindex{2D}\,\,\,] In contrast to the 1D and 3D cases, a 2D
  atmosphere extends only inside a plane. A spherical coordinate
  system is accordingly not needed and a polar system\index{polar
    coordinate system}, consisting of a radial and an angular
  coordinate, is used. The 2D case is most likely used for satellite
  measurements where the atmosphere is observed inside the orbit
  plane. The angular coordinate corresponds then to the angular
  distance along the satellite track, but the coordinate is for
  simplicity denoted as the latitude. The zero point for the 2D
  latitude is arbitrary. No lower and upper limit exists for the 2D
  latitude, and this allows that measurements from several subsequent
  orbits can be simulated as one unit. The atmosphere is treated to be
  undefined outside the considered plane. A 2D atmosphere is shown in
  Figure~\ref{fig:fm_defs:2d}.

\end{description}

 \begin{figure}[!t]
  \begin{center}
   \includegraphics*[width=0.95\hsize]{Figs/fm_definitions/atm_dim_1d}
   \caption{A schematic cross-section of a 1D atmosphere. The atmosphere is 
     here spherically symmetric. This means that the radius of the
     geoid, the ground and all the pressure surfaces is constant. The
     cloud box extends here from the ground up to the thin solid line
     (no blackbody ground assumed). The upper limit of the cloud box
     coincides always with a pressure surface. }
   \label{fig:fm_defs:1d}  
  \end{center}
 \end{figure}
 % This figure was produced by the Matlab function mkfigs_atm_dims.

 \begin{figure}[!t]
  \begin{center}
   \includegraphics*[width=0.95\hsize]{Figs/fm_definitions/atm_dim_2d}
   \caption{ Schematic of a 2D atmosphere. The radii of the geoid, the ground
     and all the pressure surfaces are for 2D atmospheres allowed to
     have a latitude variation. The limits of the cloud box coincide
     always with grid box boundaries. A blackbody ground is assumed
     here as the cloud box does not extend down to the ground.
     Plotting symbols as in Figure~\ref{fig:fm_defs:1d}. }
   \label{fig:fm_defs:2d}
  \end{center}
 \end{figure}
 % This figure was produced by the Matlab function mkfigs_atm_dims.


\levelc{Altitude coordinates}
%===================
\label{sec:fm_defs:altitudes}

\begin{description}
  
\item[Pressure\index{pressure}] The main altitude coordinate is
  pressure. This is most clearly manifested by the fact that the
  vertical atmospheric grid consists of surfaces with equal pressure.
  The vertical grid is consistently denoted as the pressure grid and
  the corresponding workspace variable is called \verb|p_grid|. The
  choice of having pressure as main altitude coordinate results in
  that atmospheric quantities are retrieved as a function pressure,
  not as a function of geometrical altitude.
  
\item[Pressure altitude\index{pressure altitude}] A basic assumption
  in ARTS is that atmospheric quantities (temperature, geometric
  altitude, species VMR etc.) vary linear with the logarithm of the
  pressure. This corresponds roughly to assuming a linear variation
  with altitude. To obtain a more intuitive unit for the logarithm of
  the pressure, the quantity pressure altitude, \PrsAlt, is
  introduced, and it is defined as
  \begin{equation}
   \PrsAlt = \aPrsAlt{s}(\log_{10}\aPrs{0} - \log_{10}\Prs)
   \label{eq:fm_defs:prsalt}
  \end{equation}
  where \aPrsAlt{s} and \aPrs{0} are arbitrary parameters (set in
  \verb|arts.h|), but suitably selected in such way that the pressure
  altitude corresponds roughly with the geometrical altitude.  This
  correspondence is achieved if the two parameters are treated as a
  fixed scale height (for decreases of the pressure with a factor of
  10) and the surface pressure, respectively. Values around
  $\aPrsAlt{s}=15.5$~km and $\aPrs{0}=1013$~hPa are suitable for
  simulations dealing with the Earth's atmosphere.
  
\item[Radius\index{radius}] Beside pressure, geometrical altitudes are
  needed to determine the propagation path through the atmosphere etc.
  The main geometrical altitude coordinate is the distance to the
  centre of the coordinate system used, the radius. This is a natural
  consequence of using a spherical or polar coordinate system. The
  radius is used inside ARTS for all geometrical calculations and to
  store the position of the sensor (Section~
  \ref{sec:fm_defs:sensor1}).
  
\item[Geometrical altitude\index{geometrical altitude}] The term
  geometrical altitude signifies here the difference in radius between
  a point and the geoid (Section~\ref{sec:fm_defs:geoid}) along the
  vector to the centre of the coordinate system (see for example
  Equation~\ref{eq:fm_defs:zground}). Hence, the geometrical altitude
  is not measured along the local zenith direction (the normal to the
  reference geoid). Geometrical altitudes are mainly used to
  facilitate the input of the ground altitude, the sensor position and
  the altitude of the pressure surfaces. This is the case as these
  quantities are known rather with respect to the geoid than with
  respect to the Earth's centre.

\end{description}


\levelc{Atmospheric grids and fields}
%===================
\label{sec:fm_defs:grids}

As mentioned above, the vertical grid of the atmosphere consists of a
set of layers with equal pressure, the pressure grid (\verb|p_grid|).
This grid must of course always be specified. It is not allowed that
there is an altitude gap between the ground and the lowermost pressure
surface.  That is, the ground pressure must be smaller than the
pressure of the lowermost vertical grid surface. On the hand, it is
not necessary to match the ground and the first pressure surface, the
pressure grid can extend below the ground level. The upper end of the
pressure grid gives the practical upper limit of the atmosphere as
vacuum is assumed above. With other words, no absorption and
refraction take place above the uppermost pressure surface.

A \qindex{latitude} grid (\verb|alpha_grid|) must be specified for 2D and 3D.
For 2D, the latitudes shall be treated as the angular distance along
the orbit track, as described above in
Section~\ref{sec:fm_defs:atmdim}.  The latitude angle is throughout
calculated for the vector going from the centre of the coordinate
system to the point of concern. Hence, the latitudes here correspond
to the definition of the geocentric latitude, and not geodetic
latitudes (see Section~\ref{sec:ppath:geoid}). This is an accordance
to the definition of geometric altitudes found above. 
For 3D, a \qindex{longitude} grid (\verb|beta_grid|) must also be specified.
Valid ranges for latitude and longitude values are given in
Section~\ref{sec:fm_defs:atmdim}. 

The atmosphere is treated to be undefined outside the latitude and
longitude ranges covered by the grids, if not the whole globe is
covered. This results in that a propagation path is not allowed to
cross a latitude or longitude end face of the atmosphere, if such
exists, it can only enter or leave the atmosphere through the top of
the atmosphere (the uppermost pressure level). See further
Sections~\ref{sec:fm_defs:atmdim} and \ref{sec:fm_defs:rte}. The
volume (or area for 2D) covered by the grids is denoted as the
\qindex{model atmosphere}.

If the longitude and latitude grids are not used for the selected
atmospheric dimensionality, then the longitude grid (for 1D and 2D)
and the latitude grid (for 1D) are set to be empty, but the when
dealing with the size of variables the grid length shall be treated to
be one ($\aVctLng{\Lat}=1$ and/or $\aVctLng{\Lon}=1$). For example,
the matrix describing the geoid (see Section~\ref{sec:fm_defs:geoid})
has for 1D the size $[1,1]$.

The basic atmospheric quantities are represented by their values at
each crossing of the involved grids (indicated by thick dots in
Figure~\ref{fig:fm_defs:2d}), or for 1D at each pressure surface
(thick dots in Figure~\ref{fig:fm_defs:1d}). This representation is
denoted as the field\index{atmospheric field} of the quantity. The
field must, at least, be specified for the geometric altitude of the
pressure surfaces (\verb|z_field|), the temperature (\verb|t_field|)
and considered atmospheric species (\verb|????|). The atmospheric
fields are tensors of order 3 with size
$[\aVctLng{\Prs},\aVctLng{\Lat},\aVctLng{\Lon}]$.

The fields are assumed to be piece-wise linear functions vertically
(with pressure altitude as the vertical coordinate,
Section~\ref{sec:fm_defs:altitudes}), and along the latitude and
longitude edges of 2D and 3D grid boxes. For points inside 2D and 3D
grid boxes, multidimensional linear interpolation is applied (that is,
bilinear interpolation for 2D etc.). Note especially that this is also
valid for the field of geometrical altitudes (\verb|z_field|).  Fields
are rank-3 tensors, for example \verb|z_field| has the dimensions
$[\aVctLng{\Prs},\aVctLng{\Lat},\aVctLng{\Lon}]$.  That means each
field is like a book, with one page for each pressure grid point, one
row for each latitude grid point, and one column for each longitude
grid point. In the 1-D case there is just one row and one column on each
page.



\levelc{The geoid and the ground}
%===================
\label{sec:fm_defs:geoid}

The \qindex{geoid}, \aRds{\odot}, is an imaginary surface used as a
reference surface when specifying the ground altitude and the altitude
of pressure surfaces. Any shape of the geoid is allowed but a smoothly
varying geoid is the natural choice, with the centres of the geoid and
the coordinate system coinciding. The geoid should normally be set to
the reference ellipsoid for some global geodetic datum, such as
WGS-84. For further reading on geoid ellipsoids and WGS-84, see
Section~\ref{sec:ppath:geoid}.

Inside ARTS, the geoid is represented as a matrix (\verb|r_geoid|),
holding the geoid radius for each crossing of the latitude and
longitude grids. The size of the matrix is accordingly $\aRds{\odot}
\SzeSmb [\aVctLng{\Lat},\aVctLng{\Lon}]$ and the geoid radius for a
specific position is $\aRds{\odot}(\aInd{\Lat},\aInd{\Lon})$. The
geoid is not defined outside the ranges covered by the latitude and
longitude grids, with the exception for 1D where the geoid by
definition is a full sphere. 
The ground altitude, \aAlt{g}, is given as the geometrical altitude
above the geoid. The radius for the ground is accordingly
\begin{equation}
  \aRds{g} = \aRds{\odot} + \aAlt{g}
 \label{eq:fm_defs:zground}
\end{equation}
As described in
Section~\ref{sec:fm_defs:grids}, it is not allowed that there is a gap
between the ground and the lowermost pressure level.

The ARTS variable for the \qindex{ground altitude}
(\verb|ground_altitude|) is a matrix of the same size as the geoid
matrix. For 1D, the ground is a sphere by definition (as the geoid),
while for 2D and 3D any shape is allowed and a rough model of the
ground topography can be made. For example, for limb sounding into the
troposphere, it could be of importance to capture the intersection of
the antenna beam by the Himalayas, and maybe other mountain ridges.
However, it should be noted that ground reflections are treated in a
simplified manner and accurate results cannot be expected beside when
some conditions are fulfilled (see
Section~\ref{sec:fm_defs:groundrefl}).

The temperature of the ground\index{ground temperature} (\aTmp{g}) is
treated to be the same as for the temperature field for the altitude
of the ground. The \qindex{ground emission} is described by the
variable \verb|ground_emissivity|, holding the ground emission factor
$(e_g)$ of the ground, as a function of frequency for each
latitude/longitude position. This workspace variable is a tensor of
order 3, with size $e_g \SzeSmb
[\aVctLng{\Frq},\aVctLng{\Lat},\aVctLng{\Lon}]$.  The emission from
the ground is $e_g(\Frq)\Planck(\Frq,\aTmp{g})$, where \Planck\ is the
Planck blackbody function. That is, the emissivity is a value between
0 and 1, where 0 means that the ground has no emissivity, and for
$e_g=1$ the ground acts as a blackbody for that particular frequency.

\begin{description}
\item[Blackbody ground\index{blackbody ground}] If the ground as a
  whole shall be treated as blackbody, which is of importance for the
  cloud box and determination of propagation paths
  (Section~\ref{sec:fm_defs:cloudbox} and \ref{sec:fm_defs:ppaths},
  respectively), all values of \verb|ground_emissivity| must be 1.
\end{description}


\levelc{The cloud box}
%===================
\label{sec:fm_defs:cloudbox}

In order to save computational time, scattering calculations are
limited as far as possible to the part of the atmosphere containing
clouds and other scattering objects (beside the ground). The
atmospheric region in which scattering shall be considered is denoted
as the \qindex{cloud box}, and it is discussed here as it acts as an
additional atmospheric limit when calculating propagation paths (see
Section~\ref{sec:fm_defs:rte}).

The cloud box is defined in such way that a propagation path entering
the cloud box at one position has not crossed the cloud box boundary
at any other location (remember that there is no scattering outside
the cloud box). This requires in general that the lower limit of the
cloud box is the ground. This is the case as if there is a gap between
the ground and the cloud box, paths entering the cloud box from below
after a ground reflection have passed through the cloud box on the way
down, at least for propagations paths close to nadir. However, if with
a blackbody ground (Section~\ref{sec:fm_defs:geoid}), there is
effectively no ground reflection
(Section~\ref{sec:fm_defs:groundrefl}) and a gap below the cloud box
is allowed. 

The cloud box is defined to be rectangular in the used coordinate
system, with limits exactly at points of the involved grids. This
means, for example, that the vertical limits of the cloud box are two
pressure surfaces. If the ground is not a blackbody, the lower limit
must be set to the lowest pressure surface (index 0). For 2D, the
angular extension of the cloud box is between two points of the
latitude grid (Figure~\ref{fig:fm_defs:2d}), and likewise for 3D but
then also with a longitude extension between two grid points.

There exists in fact a small risk for 2D and 3D, and with a reflecting
ground, that a propagation path into the cloud box has earlier crossed
the box boundaries. This can happen for propagation paths close to the
nadir direction and if the ground is tilted in such way that the
surface of the ground points towards the cloud box. The first crossing
of the cloud box is neglected for such cases, the cloud box is simply
turned off when determining the radiation field going into the cloud
box.


\levelb{Absorption and refractive index}
%==============================================================================
\label{sec:fm_defs:absorption}

[* Considerations for the calculation of absorption and refractive
index ...  Tags, line-by-line, absorption models etc. The old \verb|f_mono|
should be called \verb|f_grid|, or? (Note that I have used the name
\verb|f_grid| above.)*]



\levelb{Compulsory sensor and data reduction variables}
%==============================================================================
\label{sec:fm_defs:sensor1}

The instrument that detects the simulated radiation is denoted as the
sensor\index{sensor, the}. The forward model is constructed in such
way that a sensor must exists. For cases when only monochromatic
pencil beam radiation is of interest, the positions and directions for
which the radiation shall be calculated are given by specifying an
imaginary sensor with infinite frequency and angular resolution. The
workspace variables for the sensor that always must be specified are
\verb|sensor_pos|, \verb|sensor_los|, \verb|sensor_pol|,
\verb|antenna_dim|, \verb|psi_block_grid| and \verb|Hb|. For cases
with 2D antenna patterns (a feature not yet implemented), the
workspace variable \verb|omega_block_grid| is also compulsory.  These
variables are presented separately below in this section.  The
discussion of sensor workspace variables is continued in
Section~\ref{sec:fm_defs:sensor2}, where it is also described how
different measurement sequences are modelled in the most practical way
(Section~\ref{sec:fm_defs:howtomeasseq}). The
Section~\ref{sec:fm_defs:rte} gives further insights in how the sensor
is treated in ARTS.


\levelc{Sensor position\index{sensor position}}
%===================
\label{sec:fm_defs:sensorpos}

The observation positions of the sensor are stored in
\verb|sensor_pos|. This is a matrix where each row corresponds to a
sensor position. The number of columns in the matrix equals the
atmospheric dimensionality (1 column for 1D etc.). The columns of the
matrix (from first to last) are radius, latitude and longitude. 
Accordingly, row $i$ of \verb|sensor_pos| for a 3D case is
$(\aRds{i},\aLat{i},\aLon{i})$. The sensor position can be set to any
value, but the resulting propagation paths (also dependent on
\verb|sensor_los|) must be valid with respect to the model atmosphere
(see Section~\ref{sec:fm_defs:rte}). An obviously incorrect choice is to
place the senor below the ground altitude.

The fact that the sensor position can be given any value results in
that the radius must be used in \verb|sensor_pos|, in contrast to
\verb|z_ground| and \verb|z_field| where the altitude above the geoid
is applied. This is the case as the sensor can for 2D and 3D be
placed outside the covered latitude and longitude ranges, thus
outside the defined geoid and the geometrical altitude is undefined.

The sensor is treated to be motionless when calculating the spectrum,
or spectra, for each given observation position. One or several
spectra can be calculated for each position as described in
Section~\ref{sec:fm_defs:seqsandblocks}.


\levelc{Line-of-sight\index{line-of-sight}}
%===================
\label{sec:fm_defs:los}

The viewing direction of the sensor, the line-of-sight, is described
with two angles, the zenith angle (\ZntAng) and the azimuth angle
(\AzmAng). The zenith angle exists for all atmospheric
dimensionalities, while the azimuthal angle is defined only for 3D.
The term line-of-sight is not only used in connection with the sensor,
it is also used to describe the local propagation direction along the
path taken by the observed radiation (Section~\ref{sec:fm_defs:ppaths}).
The zenith and azimuthal angles are defined identical in these two
contexts. This is expected as the position of the sensor is the end
point of the propagation path. The line-of-sight of propagation paths
is defined in the direction a photon travels to reach the sensor,
while the sensor line-of-sight is the direction the antenna is pointed
to receive the photons. This means that the sensor and path
line-of-sights (at the sensor) are parallel but go in opposite
directions. As a true sensor has a limited spatial resolution
(described by the antenna pattern), theoretically there are an
infinite number of line-of-sights associated with the sensor, but in
the forward model spectra are only calculated for a discrete set of
directions. If a sensor line-of-sight is mentioned without any
comments, it refers to the direction in which the centre of the
antenna pattern is directed.

The \qindex{zenith angle}, \ZntAng, is simply the angle between the
line-of-sight and the zenith direction. It should be mentioned that
the zenith and nadir directions are here defined to be along the line
passing the centre of the coordinate system and the point of concern
(Section~\ref{sec:ppath:geoid}). A nadir observation,
$\ZntAng=180\degree$, is thus a measurement towards the centre of the
coordinate system. The maximum absolute value of a zenith angle is
180\degree. For 1D and 3D, positive and negative zenith angles are
treated identically, while for 2D a distinction is made. In the case
of 2D, positive and negative zenith angles mean that the viewing
direction is towards higher and lower latitudes, respectively.

The \qindex{azimuthal angle}, \AzmAng, is given with respect to the
plane going through the north and south poles of the coordinate system
$(\Lat=\pm90\degree)$ and the sensor. The valid range is
$[-180\degree,180\degree]$ where angles are counted clockwise and
0\degree means that the viewing or propagation direction is north-wise.
Hence, +90\degree means that the direction of concern goes eastward.

The line-of-sight for propagation paths is discussed further in
Section~\ref{sec:fm_defs:ppaths}. The sensor line-of-sights are stored in
\verb|sensor_los|. This workspace variable is a matrix, where the
first column holds zenith angles and the second column is azimuthal
angles. For 1D and 2D there is only one column in the matrix, while
for 3D a row $i$ of the matrix is $(\aZntAng{i},\aAzmAng{i})$. The
number of rows for \verb|sensor_los| must be the same as for
\verb|sensor_pos|.


\levelc{Sensor polarisation\index{sensor polarisation}}
%===================
\label{sec:fm_defs:sensorpol}

[* How shall this be handled? Can a transfer matrix be used?
\verb|sensor_pol| *]


\levelc{Sensor characteristics and data reduction}
%===================
\label{sec:fm_defs:sensorchar}

Sensor characteristics\index{sensor characteristics} is used here as a
comprehensive term for the response of all sensor parts, beside
polarisation effects, that affect how the field of monochromatic
pencil beam intensities are translated to the recorded spectrum. For
example, the antenna pattern, the side-band filtering and response of
the spectrometer channels are normally the most important
characteristics for a microwave heterodyne radiometer. Any processing
of the spectral data that takes place before the retrieval is denoted
as \qindex{data reduction}. The most common processing is to represent
the original spectra with a smaller set of values, that is, a
reduction of the data size. The most common data reduction techniques
is binning and Hotelling transformation by an eigenvector expansion
(Section~[**]).

The influence of sensor characteristics and data reduction is in ARTS
incorporated by transfer matrices\index{sensor transfer matrix}, as
described in Section~\ref{sec:formalism:sensor}. The application of
these transfer matrices assumes that each step is a linear operation,
which should be the case for the response of the parts of a well
designed instrument. Non-linear data reduction is handled by special
workspace functions.

The creation of sensor and data reduction transfer matrices is not
(yet?) included in ARTS and no workspace variables exist to describe
the individual sensor parts etc. The sensor and data reduction are
described as a series of units, each having its own transfer matrix.
There is only one compulsory transfer matrix and it is \aSnsMtr{b}
(\verb|Hb|), where the subscript $b$ stands for block (see further
Section~\ref{sec:fm_defs:seqsandblocks}). There are several workspace
variables associated with this transfer matrix where
\verb|antenna_dim|, \verb|psi_block_grid| and \verb|omega_block_grid|
are the possibly compulsory ones.

The variable \verb|antenna_dim| gives the dimensionality of the
antenna pattern\index{antenna pattern dimensionality}, where the
options are 1 and 2, standing for 1D and 2D, respectively. A 1D antenna
dimensionality means that the azimuthal extension of the antenna
pattern is neglected, there is only a zenith angle variation of the
response. A 2D antenna pattern is converted to a 1D pattern by
integrating the azimuthal response for each zenith angle. ARTS handles
not yet 2D antenna patterns and the only allowed choice for
\verb|antenna_dim| is 1. The presentation below assumes that the 2D
option is implemented.

For each sensor position, a number of monochromatic pencil beam
spectra are calculated. The monochromatic frequencies are given by
\verb|f_grid| (Section~\ref{sec:fm_defs:absorption}). The pencil beam
directions are obtained by summing the line-of-sight angles
(\verb|sensor_los|) for the position and the values of
\verb|psi_block_grid| and \verb|omega_block_grid|. For example, pencil
beam zenith angle $i$ is calculated as
\begin{equation}
  \aZntAng{i} = \aZntAng{0} + \Delta\aZntAng{i}
  \label{eq:fm_defs:psi_grid}
\end{equation}
where \aZntAng{0} is the sensor line-of-sight for the position and
$\Delta\aZntAng{i}$ is value $i$ of \verb|psi_block_grid|.  With other
words, \verb|psi_block_grid| and \verb|omega_block_grid| are the
calculation grids to use when the intensity field is weighted with the
antenna response. For 1D antenna patterns, \verb|omega_block_grid| is
assumed to be a vector of length 1 with the value 0 (any existing
values are ignored). 


\levelc{Measurement sequences and blocks}
%===================
\label{sec:fm_defs:seqsandblocks}

The series of observations modelled by the simulations is denoted as
the \qindex{measurement sequence}. That is, a measurement sequence covers all
spectra recorded at all considered sensor positions
(\verb|sensor_pos|). The observations performed at one position is
referred to as a \qindex{measurement block}. 

A measurement block corresponds to one or several recorded spectra,
depending on the measurement conditions and the atmospheric
dimensionality. A block can consists of several spectra when there is
no effective motion of the sensor with respect to the atmospheric
fields. It should be noted that for 1D cases, a motion along a
constant radius has no influence on the simulated spectra as the same
atmospheric fields are seen for a given viewing direction. It is
normally favourable, if possible, to handle all spectra as a single
block, instead of using a block for each sensor position. This is the
case as the antenna patterns for the different line-of-sights are
normally overlapping and a pencil beam spectrum can be used in
connection with several measurement spectra. If a measurement sequence
is divided into several blocks even if a single block would be
sufficient, pencil beam spectra for close line-of-sights can be
calculated several times, which of course if unnecessary.  To
summerise, for cases when the sensor is not in motion, or with a 1D
atmosphere and a sensor not moving vertically, the aim should be to
use a single block for the measurement sequence. This discussion is
continued in Section~\ref{sec:fm_defs:howtomeasseq}.

For each block, pencil beam spectra are calculated for the
line-of-sights obtained when summing \verb|sensor_los| and
\verb|psi_block_grid| (and possibly \verb|omega_block_grid|), as
described in Section~\ref{sec:fm_defs:sensorchar}. The pencil beam
spectra for each line-of-sight are appended vertically to form a
common vector, \aMpiVct{b}. Values are put in following the order
in \verb|f_grid|. Hence, the frequencies for this vector are
\begin{equation}
  \aMpiVct{b} = 
  \left[ \begin{array}{c} 
     \left[
          \begin{array}{c} \aFrq{1}\\\vdots\\\aFrq{n} \end{array} 
     \right] \\
     \vdots \\
     \left[
          \begin{array}{c} \aFrq{1}\\\vdots\\\aFrq{n} \end{array} 
     \right]
     \end{array} \right]
  \label{eq:fm_defs:freqs_of_ib}
\end{equation}
where \aFrq{i} is element $i$ of \verb|f_grid| and $n$ the length of
the same vector. The order of the angles inside \verb|psi_block_grid|
and \verb|omega_block_grid| is followed when looping the pencil beam
directions, where the zenith angle direction is the innermost loop.
That is, for 2D antenna patterns all zenith angles are looped for the
first azimuthal angle etc.

The transfer matrix \verb|Hb| is applied on each \aMpiVct{b} and the
results are appended vertically, following the order of the positions
in \verb|sensor_pos|:
\begin{equation}
  \MsrVct = \left[ \begin{array}{c} \aSnsMtr{b}\aMpiVct{{b,1}} \\ 
                                    \aSnsMtr{b}\aMpiVct{{b,2}} \\
                                    \vdots                  \\
                                    \aSnsMtr{b}\aMpiVct{{b,n}} 
            \end{array} \right]
  \label{eq:fm_defs:measseq}
\end{equation}
where \MsrVct\ is defined in Equation~\ref{eq:formalism:fm} and $1$
indicates the first sensor position etc. This equation shows that
\verb|Hb| shall contain at least a description of the antenna
response. The matrix \verb|Hb| can also cover other sensor
characteristics and data reduction if the features of concern are
common for all measurement blocks. If this is not the case, those
features must be handled by the other existing transfer matrices or by
other means, as described in Section~\ref{sec:fm_defs:howtomeasseq}.

It should be noted that the compulsory sensor variables give no
information about the content of the obtained \MsrVct, as it is not
clear which parts and features the block transfer matrix covers. If
\verb|Hb| only incorporates the antenna pattern, the result is a set
of hypothetical spectra corresponding to a point inside the sensor. On
the other hand, if \verb|Hb| includes the whole of the sensor and an
eigenvector data reduction, the result is not even a spectrum is
traditional way, it is just a row of coefficients with a vague
physical meaning. To handle quantities appearing at a stage between
monochromatic pencil beam intensities and the final representation of
the measurements (where the total transformation can involve more
steps than the multiplication with \verb|Hb|), additional sensor
variables must be introduced, as described in
Section~\ref{sec:fm_defs:sensor2}.


\levelb{Clear sky radiative transfer}
%==============================================================================
\label{sec:fm_defs:rte}

\newcommand{\Int}{{\bf I}}
\newcommand{\Ext}{{\bf K}}
\newcommand{\Abs}{{\bf a}}
\newcommand{\Sca}{{\bf Y}}
\newcommand{\Dir}{{\bf n}}
\newcommand{\Path}{{\bf s}}
%\newcommand{\Planck}{{\bf B}}
\newcommand{\Freq}{\nu}
\newcommand{\Dep}{(\Dir,\Freq)}

\startsymbols
  \Ind           & -                 & vector/matrix/tensor index           \\
  \Int           & 
 \label{symtable:fm_defs_rt}     
\stopsymbols

% STEFAN, you should have a look at the file symbol_defs. The quantities for
% which you have defined commands should be of general interest and they should
% then be included in that file. For example, there are already macros for
% frequency (\Frq and \aFrq{i}).
%
% I see that you use a capital I for the intensity Stokes vector. I have so
% far (and I don't think there are any exceptions AUG) used lower case letters
% for vectors and upper case for matrices. I would be happy if we could stick
% to this rule, which I think is a standard way to keep matrices and vectors
% separated. Note the macro \MpiVct in symbol_defs.
%
% I just put in a macro for Planck, but then the scalar version. The Planck
% function here is really not a scalar? If not, maybe you can introduce the
% macro \VctPlanck? But what symbol should be used? Lower case b is already
% occupied.
%
% I also think that you should put ''your'' symbols in the table found at the
% start of the section as it is valid for the whole chapter. 
%
% Patrick

\levelc{Radiative transfer equation and important quantities}

The radiative transfer equation in the absence of cloud particles is
\citep{mishchenko00:_light_scatt_nonsp_partic}: 

\begin{equation}
  \label{eq:rte_4stokes_no_scat}
  \frac{d\Int\Dep}{d\Path} = - \Ext\Dep \Int\Dep + \Abs\Dep\Planck(T)
\end{equation}
where the meaning of the quantities in this equation is as follows:

The specific intensity vector $\Int$



 where $I$ is the monochromatic pencil beam intensity, $l$ distance
 along the line of sight (LOS), $l_1$ the point of the considered part
 of the LOS furthest away from the sensor, $l_2$ the closest point of
 the LOS, $I_1$ the intensity at $l_1$, $\kappa$ the total absorption
 along the LOS and $\sigma$ the source function.




\levelc{Calculation procedure}
%===================
\label{sec:fm_defs:calcproc}

The overall structure of the part solving the radiative transfer
equation is fixed. The corresponding workspace function is
\verb|rteCalc|. The calculation procedure of \verb|rteCalc| is
outlined in Algorithm~\ref{alg:fm_defs:rteCalc}. For further details
of each calculation step, see the indicated equation or section.

The function \verb|rteCalc|, beside providing atmospheric spectra,
performs also the calculation of weighting functions for atmospheric
variables where analytical expressions are used. These calculations
follow the same outline but are not included here for simplicity, they
are instead considered in Section~\ref{sec:fm_defs:wfs}. 

\begin{algorithm}
 \begin{algorithmic}
  \IF[{Section \ref{sec:fm_defs:scattering}}]{cloud box is turned on}
   \STATE{determine the intensity field inside the cloud box}
  \ENDIF
  \STATE{allocate memory for the measurement vector, \MsrVct}
  \COMMENT{Equation \ref{eq:formalism:fm}}
  \FOR{each sensor position}
   \STATE{allocate memory for the vector \aMpiVct{b}}
   \COMMENT{Equation \ref{eq:fm_defs:freqs_of_ib}}
   \FOR[Section \ref{sec:fm_defs:seqsandblocks}]
                                    {each pencil beam direction of the block}
    \STATE{determine the propagation path}
    \COMMENT{Section \ref{sec:fm_defs:ppaths}}
    \STATE{initialise the vector \MpiVct\ with the radiative background}
    \COMMENT{Section \ref{sec:fm_defs:ppaths}}
    \FOR{each step along propagation path}
     \STATE{update \MpiVct\ following Equation [**]}
     \IF{at ground reflection}
      \STATE{update \MpiVct\ following Equation [**]}
     \ENDIF
    \ENDFOR
    \STATE{copy \MpiVct\ to correct part of \aMpiVct{b}}
   \ENDFOR
   \STATE{put the product \aSnsMtr{b}\aMpiVct{b} in correct part of \MsrVct}
  \ENDFOR
 \end{algorithmic}
 \caption{Outline of the clear sky radiative transfer calculations. An emission
   measurement, without any weighting function calculations, is
   assumed here. Optical thicknesses and weighting functions are
   calculated following the same scheme.}
 \label{alg:fm_defs:rteCalc}
\end{algorithm}


\levelc{Propagation paths}
%===================
\label{sec:fm_defs:ppaths}

A pencil beam path through the atmosphere to reach a specified
position from a specified line-of-sight is denoted as a propagation
path. The forward direction of the propgations paths is towards the
position for which the radiative intensity will be calculated. It
should be noted that with scattering there exist an infinite number
of propgation paths for a given combination of observation point and
line-of-sight. 

Propagation paths are described by storing a set of points on the
path, and the distance along the path between the points. These
quantities, and a number of auxilary variables, are stored together in
a structure described in Section~[**]. The path points are primarily
placed at the crossings of the path with the atmospheric grids
(\verb|p_grid|, \verb|alpha_grid| and \verb|beta_grid|). A path point
is also placed at the sensor if it is placed inside the atmosphere.
Ground reflections and tangent points are also included if usch exist.
There exists also the possibility to set an upper limit for the
distance along the path between the points by the workspace variable
\verb|ppath_lmax|. If \verb|ppath_lmax| is set to a value $\leq 0$ then
no points are added to the path, otherwise points are included to
fulfill the length criterion. 

The propagation paths are determined basically by starting at the
sensor and following the path backwards, by geometrical calculations
or some ray tracing technique depending on if refraction is considered
or not, until the other end point is found. If the sensor is placed
above the model atmosphere, geometrical calculations are used (as
there is no refraction in space) to find the crossing between the path
and the top of the atmosphere where the ray tracing then starts.  If
there is no cloud box and the ground not acts as a blackbody, the
starting point of the propagation paths is always found at the top of
the atmosphere (where the path enters the atmosphere). Propagation
paths are not followed inside cloud boxes and the starting point is
set to the cloud box boundary. If there is an intersection with path
of a blackbody ground (Section~\ref{sec:fm_defs:geoid}), the
propagation path is considered to start at the ground. This is the
case as a blackbody ground absorbs all incoming radiation and the
radiative transfer along the path before the ground reflection is of
no interest. Example on propagation paths are shown in
Figure~\ref{fig:fm_defs:ppath_cases1}.

The radiative intensity at the starting point of the path, and in the
direction of the line-of-sight at that point, is denoted as the
radiative background. The radiative background at the top of the
atmosphere is given by the workspace variable \verb|i_space|. This
variable should normally be set to cosmic background radiation, if not
the sensor is directed towards the sun. Remaining possible radiative
backgrounds are blackbody radiation from the ground or the outgoing
intensity field at the boundary of a cloud box.

 \begin{figure}[!t]
  \begin{center}
   \includegraphics*[width=0.95\hsize]{Figs/fm_definitions/ppath_cases1}
   \caption{Propagation path examples for a 2D atmosphere. The atmosphere 
     and the cloud box are plotted as in Figure~\ref{fig:fm_defs:2d}
     beside that the points for the atmospheric fields are not
     emphesised. For the propagation paths to the left, with a the
     sensor placed inside the model atmosphere, a reflecting ground is
     assumed. The paths to the right are valid for a sensor placed
     outside the atmosphere (at the $*$) and a blackbody ground. No
     value for the maximum path step length was applied. Note the tangent
     point inserted for the uppermost propagation path, plotted as $\oplus$.}
   \label{fig:fm_defs:ppath_cases1}
  \end{center}
 \end{figure}
 % This figure was produced by the Matlab function mkfigs_ppath_cases.


\levelc{Solving the radiative transfer equation}
%===================
\label{sec:fm_defs:solverte}


\levelc{Ground reflections}
%===================
\label{sec:fm_defs:groundrefl}

The ground is treated to be a totally flat surface; incoming radiation
is only reflected in a single direction, with the normal assumption
that the angle to the surface normal is equal for the incoming and
outgoing line-of-sight. If there is slope of the ground (a variation
of the ground radius), this slope is considered when determining the
line-of-sight for the incoming radiation (remember that propagation
paths are followed backwards). This is exemplified in
Figure~\ref{fig:fm_defs:ppath_cases2}, for a case with a very strong
ground slope.

 \begin{figure}[!t]
  \begin{center}
   \includegraphics*[width=0.95\hsize]{Figs/fm_definitions/ppath_cases2}
   \caption{The influence of a ground slope on the propagation path. The zenith
     angle of the sensor line-of-sight is the same for the two paths (but with
     different sign, which matters here as this is a 2D case), and the paths
     would have been symmetric around the latitude of the sensor without a 
     ground slope. The maximum path step length is here set to be a value
     equaling half the vertical distance between the two shown pressure
     surfaces.}
   \label{fig:fm_defs:ppath_cases2}
  \end{center}
 \end{figure}
 % This figure was produced by the Matlab function mkfigs_ppath_cases.

When solving the radiative transfer equation, a ground reflection is
included as
\begin{equation}
  \Mpi(\Frq) \gets \Mpi(\Frq)(1-e_g(\Frq)) + e_g(\Frq)B(\Frq,\aTmp{g})
  \label{eq:fm_defs_ground_reflection}
\end{equation}
where $e_g$ and \aTmp{g} is the ground emissivity and temperature,
respectively, for the point where the ground reflection takes place.
Note that the ground emissivity can vary both with frequency and
position (Section~\ref{sec:fm_defs:geoid}).
Equation~\ref{eq:fm_defs_ground_reflection} shows that the path before
a reflection with a blackbody ground can be neglected. For such cases,
$\Mpi(\Frq)$ equals $B(\Frq,\aTmp{g})$ independently of the prevoius
value of $\Mpi(\Frq)$. The ground emission is treated to be
unpolarised.

A more detailed treatment of ground intersections would take into
account the scattering properties of the ground and this limitation of
the simulations should be considered. However, a requirement for that
the recorded spectra shall be affected considerably by the ground
properties is that the optical thickness of the lower troposphere is
not too high. If it is too high, the radiation is absorbed before it
reaches the sensor. For moderate values on the optical thickness (in
rough numbers, $1.5\leq\tau\leq4$), when some radiation can propagate
from the ground to the sensor, it should be noted that the intensity
field (as a function of zenith angle) reaching the ground should be
relatively homogenous and the negliance of ground scattering will not
influence the accuracy of the results too seroiusly. With a high or
moderate optical thickness of the lower troposphere, the emission
reaching the ground will have brightness temperature close to the
physical temperatures of the lower troposphere, with a relatively low
angular variation. [* Does anybody know of important the scattering of
the ground is for nadir observations? For example, how large is the
scattering of the sea surface? *]



\levelb{Additional sensor and data reduction variables} 
%===================
\label{sec:fm_defs:sensor2}

\verb|antenna_psi_grid|, \aSnsMtr{b}, \aSnsMtr{c}


\levelc{How to model different measurement sequences in best way?}
%===================
\label{sec:fm_defs:howtomeasseq}

[* Discuss e.g. the relationship between \verb|sensor_los| and the block LOS grids. *]


\levelb{Scattering}
%==============================================================================
\label{sec:fm_defs:scattering}

[* Special quantities and considerations for scattering ... *]



\levelb{Weighting functions}
%==============================================================================
\label{sec:fm_defs:wfs}

[* This part is still quite unclear. However, the analytical
atmospheric WFs will be calculated on the same time as spectra. We
need to find a good way to specify the WFs which is a complex thing to
do. *]


%%% Local Variables: 
%%% mode: latex
%%% TeX-master: "uguide"
%%% End: 
