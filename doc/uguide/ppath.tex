\chapter{Propagation paths}
 \label{sec:ppath}


\starthistory
  120202 & Revised and parts moved to \theory\ (Patrick Eriksson).\\
  030310 & First complete version written by Patrick Eriksson.\\
\stophistory


\graphicspath{{Figs/ppath/}}


A propagation path is the name given in ARTS to the way the radiation travels
to reach the sensor for a specified line-of-sight. Propagation paths are
introduced in Section \ref{sec:fm_defs:ppaths} and this section provides
further details. For a general usage of ARTS, it should suffice to read
Section~\ref{sec:ppath:usage}. The remaining sub-sections deal with more
low-level aspects of the calculations, and are of interest only if you want to
understand the finer details of ARTS. The actual equations used are found in
Chapter~\ref{T-sec:ppaththeory} of \theory.



\section{Practical usage}
%===================
\label{sec:ppath:usage}

The ray tracing algorithm to be applied for the calculation of propagation path
is effectively selected by specifying \wsaindex{ppath\_step\_agenda} (see
further Section~\ref{sec:fm_defs:ppaths}). The fastest calculations are
obtained if refraction is neglected, denoted as geometrical calcutions. The
workspace method to apply if this assumption can be made is
\wsmindex{ppath\_stepGeometric}.

The main consideration for using \builtindoc{ppath\_stepGeometric} is to select
a value for \wsvindex{ppath\_lmax}. This variable controls to some extent the
calculation accuarcy, as described in Section~\ref{sec:fm_defs:accuracy}. This
variable sets the maximum distance between points of the (final) propagation
path. Set this variable to e.g.\ -1 if you don't want to apply such a length
criterion.

A straightforward, but inefficient, treatment of refraction is provided by
\wsmindex{ppath\_stepRefractionEuler}. This method divides the propagation path
into a series of geomtrical ray tracing steps. The size of the ray tracing
steps is selected by \wsvindex{ppath\_lraytrace}. This variable affects only
the ray tracing part, the distance between points of the propagation path
actually returned is controled by \builtindoc{ppath\_lmax} as above.





\section{Calculation approach}
%===================
\label{sec:ppath:approach}

The propagation paths are calculated in steps, as outlined in
Section~\ref{sec:fm_defs:ppaths}. The path steps are normally from one crossing
of the atmospheric grids to next. To introduce
propagation paths steps was necessary to handle the iterative solution for
scattering inside the cloud box, as made clear from Figure
\ref{fig:scattering:average}.

A full propagation path is stored in the workspace variable \wsvindex{ppath},
that is of the type \builtindoc{Ppath} (see Section \ref{sec:ppath:Ppath}). The
paths are determined by calculating a number of path steps. A path step is the
path from a point to the next crossing of either the pressure, latitude or
longitude grid (Figure~\ref{fig:ppath:ex1}). There is one
exception to this definition of a path step, and that is when there is an
intersection with the surface, which ends the propagation path at that point.
The starting point for the calculation of a path step is normally a grid
crossing point, but can also be an arbitrary point inside the atmosphere, such
as the sensor position. Only points inside the model atmosphere are handled.
The path steps are stored in the workspace variable \wsvindex{ppath\_step},
that is of the same type as \builtindoc{ppath}.

\begin{figure}
 \begin{center}
  \includegraphics*[width=0.80\hsize]{ppath_ex1}
  \caption{Tracking of propagation paths. For legend, see 
    Figure \ref{fig:ppath:ex2}. The figure tries to visualize how the
    calculations of propagation paths are performed from one grid cell
    to next. In this example, the calculations start directly at the
    sensor position $(\ast)$ as it placed inside the model
    atmosphere. The circles give the points defining the propagation
    path. Path points are always included at the crossings of the grid
    cell boundaries. Such a point is then used as the starting point
    for the calculations inside the next grid cell. }
  \label{fig:ppath:ex1}  
 \end{center}
\end{figure}
% This figure was produced by the Matlab function mkfigs_ppath

\begin{figure}
 \begin{center}
   \includegraphics*[width=0.98\hsize]{ppath_ex2}
  \caption{As Figure \ref{fig:ppath:ex1}, but with a length criterion 
    for the distance between the points defining the path.
    The inclusion of the tangent point is not a result of this length
    criterion, it is always included among the path points.}
  \label{fig:ppath:ex2}  
 \end{center}
\end{figure}
% This figure was produced by the Matlab function mkfigs_ppath


Propagation paths are calculated with the internal function
\funcindex{ppath\_calc}. The communication between this method and
\builtindoc{ppath\_step\_agenda} is handled by \builtindoc{ppath\_step}.
That variable is used both as input and output to
\builtindoc{ppath\_step\_agenda}.  The agenda gets back
\builtindoc{ppath\_step} as returned to \builtindoc{ppath\_calc} and the
last path point hold by the structure is accordingly the starting
point for the new calculations. If a total propagation path shall be
determined, the agenda is called repeatedly until the starting point
of the propagation path is found and \builtindoc{ppath\_step} will hold
all path steps that together make up \builtindoc{ppath}. The starting
point is included in the returned structure.

The path is determined by starting at the end point and moving
backwards to the starting point. The calculations are initiated by
filling \builtindoc{ppath\_step} with the practical end point of the
path. This is either the position of the sensor (true or
hypothetical), or some point at the top of the atmosphere (determined
by geometrical calculations starting at the sensor). This
initialization is not handled by \builtindoc{ppath\_step\_agenda}. 
The field \shortcode{constant} is set by \builtindoc{ppath\_calc}
to the correct value if the sensor is above the model atmosphere.
Otherwise, the field is set to be negative and is corrected by
\builtindoc{ppath\_step\_agenda} at the first call. This procedure is
needed as the propagation path constant changes if refraction is
considered, or not, when the sensor is placed inside the atmosphere.

The agenda performs only calculations to next crossing of a grid, all
other tasks are performed by \builtindoc{ppath\_calc}, with one exception.
If there is an intersection with the surface, the calculations stop at
this point. This is flagged by setting the background field of
\builtindoc{ppath\_step}. Beside this, \builtindoc{ppath\_calc} checks if
the starting point of the calculations is inside the scattering box or
below the surface level, and check if the last point of the path has
been reached. 

In many cases the propagation path can/must be considered to consist
of several parts. One exemple is surface reflection (see
Figure \ref{fig:fm_defs:surface_refl}). The variable \builtindoc{ppath}
describes then only a single part of the propagation path.



\section{The propagation path data structure}
%===================
\label{sec:ppath:Ppath}

A propagation path is represented by a structure of type
\typeindex{Ppath}. This structure holds also auxiliary variables to
facilitate the radiative transfer calculations and to speed up the
interpolation. The fields of \builtindoc{Ppath} are described below,
where the data type is given inside square brackets. 

\begin{description}

  \item[dim] [Index] The atmospheric dimensionality. This field shall always 
     be equal to the workspace variable \builtindoc{atmosphere\_dim}.
     
   \item[np] [Index] Number of positions to define the propagation
     path. Allowed values are $\geq 0$. The number of rows of
     \shortcode{pos} and \shortcode{los}, and the length of
     \shortcode{z}, \shortcode{gp\_p}, \shortcode{gp\_lat} and
     \shortcode{gp\_lon}, shall be equal to \shortcode{np}. The length
     of \shortcode{l\_step} is \shortcode{np} - 1. If \shortcode{np}
     $\leq$ 1, the observed spectrum is identical to the radiative
     background. For cases where the sensor is placed inside the model
     atmosphere and \shortcode{np} = 1, the stored position is
     identical to the sensor position and that position can be used to
     determinate the radiative background (see below).

   \item[constant] [Numeric] The propagation path constant. Such a
     constant can be assigned to all geometrical paths and for 1D
     cases (with or without refraction). This field can be
     initiated to a negative value to indicate that the constant is
     undefined or not yet set. For cases where the constant applies,
     \builtindoc{ppath\_step\_agenda} sets this constant at the first
     call of the agenda if the given value is negative.

   \item[pos] [Matrix] The position of the propagation path points.
     This matrix has \shortcode{np} rows and up to 3 columns. Each row
     holds a position where column 1 is the radius, column 2 the
     latitude and column 3 the longitude (cf.
     Section \ref{sec:fm_defs:sensorpos}). The number of columns for
     1D and 2D is 2, while for 3D it is 3. The latitudes are stored
     for 1D cases as these can be of interest for some applications
     and are useful if the propagation path shall be plotted. The
     latitudes for 1D give the angular distance to the sensor (see
     further Section \ref{sec:fm_defs:atmdim}).
     The propagation path is stored in reversed order, that is, the
     position with index 0 is the path point closest to the sensor
     (and equals the sensor position if it is inside the atmosphere).
     The full path is stored also for 1D cases with symmetry around a
     tangent point (in contrast to ARTS-1). 
     
  \item[z] [Vector] The geometrical altitude for each path position. The
     length of this vector is accordingly \shortcode{np}. This is a help
     variable for plotting and similar purposes. It shall not be used to
     interpolate the atmospheric fields, as pressure is the main altitude
     coordinate.
     
   \item[l\_step] [Vector] The length along the propagation path
     between the positions in \shortcode{pos}. The first value is the
     length between the first and second point etc. For \shortcode{np}
     $\geq 2$, the length of the vector is \shortcode{np} - 1.
     Otherwise it is 0.

   \item[gp\_p] [ArrayOfGridPos] Index position with respect to the
     pressure grid. The structure for grid positions is described in
     \developer, Section \ref{D-sec:interpolation:gridpos}. 
     
   \item[gp\_lat] [ArrayOfGridPos] As \shortcode{gp\_p} but with
     respect to the latitude grid.

   \item[gp\_lon] [ArrayOfGridPos] As \shortcode{gp\_p} but with
     respect to the longitude grid.
     
   \item[los] [Matrix] The line-of-sight of the propagation path at
     each point. The number of rows of the matrix is \shortcode{np}.
     For 1D and 2D, the matrix has a single column holding the zenith
     angle. For 3D there is an additional column giving the azimuth
     angle. The zenith and azimuth angles are defined in
     Section \ref{sec:fm_defs:los}. If the radiative background is the
     cloud box, the last position (in \shortcode{pos}) and
     line-of-sight give the relevant information needed when
     extracting the radiative background from the cloud box intensity
     field.
     
   \item[background] [String] The radiative background for the
     propagation path. The possible
     options for this field are 'space', 'blackbody surface', 'cloud
     box interior' and 'cloud box surface', where the source of
     radiation should be clear the content of the strings.
     
   \item[tan\_pos] [Vector] The position of the tangent point. This
     vector is only set if there exists a tangent point (above the
     surface level), the length of the vector is otherwise 0. The
     tangent point is defined as the point with the lowest radius
     along the path. This means that (the absolute value of) the
     zenith angle at the tangent point is always 90\degree. For 2D
     and 3D this point can deviate from the point with lowest
     geometrical altitude.
     
   \item[geom\_tan\_pos] [Vector] The position of the geometrical
     tangent point. This vector is set for all downward observations.
     Refraction and surface reflections are neglected when calculating
     this tangent point position. This field is not handled by
     \builtindoc{ppath\_step\_agenda}. Definition of the tangent point
     as for \shortcode{tan\_pos}.

   \item[nreal] [Vector] The real part of the refractive index at each path
     position. Length is accordingly \shortcode{np}. 

\end{description}




\section{Structure of implementation}
%===================
\label{sec:ppath:structure}

The workspace method for calculating propagation paths is
\wsmindex{ppathCalc}, but this is just a getaway function for
\builtindoc{ppath\_calc}. The main use of \builtindoc{ppathCalc} is to
debug and test the path calculations, and that WSM should normally not
be part of the control file. Propagation paths, or steps, are
generated from inside other functions.


\subsection{Main functions for clear sky paths}
%===================

The master function to calculate full clear sky propagation paths is
\funcindex{ppath\_calc}. This function is outlined in
Algorithm \ref{alg:ppath:ppath_calc}. The function can be divided into
three main parts, initialization (handled by
\builtindoc{ppath\_start\_stepping}), a repeated call of
\builtindoc{ppath\_step\_agenda} and putting data into the return
structure (\builtindoc{ppath}).

\begin{algorithm}
 \begin{algorithmic}
  \STATE{check consistency of function input}
  \STATE{call \builtindoc{ppath\_start\_stepping} to set 
    \builtindoc{ppath\_step}}
  \STATE{create an array of \builtindoc{Ppath} structures, 
         \builtindoc{ppath\_array}}
  \STATE{add \builtindoc{ppath\_step} to \builtindoc{ppath\_array}}
  \WHILE{radiative background not reached}
   \STATE{call \wsaindex{ppath\_step\_agenda}}
   \IF{path is at the highest pressure surface}
    \STATE{radiative background is space}
   \ELSIF{path is at either end point of latitude or longitude grid}
    \STATE{this is not allowed, issue an runtime error}
   \ENDIF
   \IF{cloud box is active}
    \IF{path is at the surface of the cloud box}
     \STATE{radiative background is the cloud box surface}
    \ENDIF
   \ENDIF
   \STATE{add \builtindoc{ppath\_step} to \builtindoc{ppath\_array}}
  \ENDWHILE
  \STATE{initialize the WSV \builtindoc{ppath} to hold found number of 
         path points} 
  \STATE{copy data from \builtindoc{ppath\_array} to \builtindoc{ppath}}
 \end{algorithmic}
 \caption{Outline of the function \builtindoc{ppath\_calc}.}
 \label{alg:ppath:ppath_calc}
\end{algorithm}

The main task of the function \funcindex{ppath\_start\_stepping} is to
set up \wsvindex{ppath\_step} for the first call of
\builtindoc{ppath\_step\_agenda}, which means that the practical
starting point for the path calculations must be determined. If the
sensor is placed inside the model atmosphere, the sensor position
gives directly the starting point. For cases when the sensor is found
outside the atmosphere, the point where the path exits the atmosphere
must be determined. The exit point can be determined by pure
geometrical calculations (see Sections \ref{sec:ppath:basicgeom} and
\ref{sec:ppath:stepcalc}) as the refractive index is assumed to have the
constant value of 1 outside the atmosphere. The problem is accordingly
to find the geometrical crossing between the limit of the atmosphere
and the sensor line-of-sight (LOS). The function performs further some
other tasks, which include:
\begin{itemize}
\item For all LOS with a zenith angle $\geq 90\degree$ the position of
  the geometrical tangent point is calculated.
\item If the sensor is placed inside the model atmosphere
  \begin{itemize}
  \item Checks that the sensor is placed above the surface level. If
    not, an error is issued.
  \item Checks for 2D and 3D and when the sensor position as at an end
    point of the latitude or longitude grid, that the LOS is inwards
    with respect to the atmospheric limit.
  \item If the sensor and surface altitudes are equal, and the sensor
    LOS is downward, the radiative background is set to be the
    surface. For 2D and 3D, the tilt of the surface radius is considered
    when determining if the LOS is downward.
  \item If the cloud box is active and the sensor position is inside
    the cloud box, the radiative backsurface is set to be ``cloud box
    interior''. All sensor positions on the cloud box surface are for
    2D and 3D treated as points inside the box (for simplicity
    reasons), while for 1D the behavior is as expected.
  \end{itemize}
\item If the sensor is placed outside the model atmosphere
  \begin{itemize}
  \item Checks that the zenith angle is $\geq  90\degree$.  Upward
    observations are here not allowed.
  \item If it is found for 2D and 3D that the exit point of the path
    not is at the top of the atmosphere, but is either at a latitude
    or longitude end face of the atmosphere, an error is issued. This
    problem can not appear for 1D.
  \end{itemize}
\end{itemize}
For further details, see the code.


\subsection{Main functions for propagation path steps}
%===================

Example on workspace methods to calculate propagation path steps are
\wsmindex{ppath\_stepGeometric} and
\wsmindex{ppath\_stepRefractionEuler}. All such methods adapt
automatically to the atmospheric dimensionality, but the different
dimensionalities are handled by separate internal functions. For
example, the sub-functions to \builtindoc{ppath\_stepGeometric} are
\funcindex{ppath\_step\_geom\_1d}, \funcindex{ppath\_step\_geom\_2d}
and \funcindex{ppath\_step\_geom\_3d}. See \shortcode{m\_ppath.cc} to
get the names of the sub-functions for other propagation path step
workspace methods. 

\begin{algorithm}
 \begin{algorithmic}
  \STATE{call \funcindex{ppath\_start\_2d}}
  \IF{\shortcode{ppath\_step.ppc} $<1$}
   \STATE{calculate the path constant}
   \COMMENT{this is then first path step}
  \ENDIF
  \STATE{call \funcindex{do\_gridcell\_2d}}
  \STATE{call \funcindex{ppath\_end\_2d}}
  \IF{calculated step ends with tangent point}
   \STATE{call \shortcode{ppath\_step\_geom\_2d} with temporary 
     \shortcode{Ppath} structure}
   \STATE{append temporary \shortcode{Ppath} structure to 
     \shortcode{ppath\_step}}
  \ENDIF
 \end{algorithmic}
 \caption{Outline of the function \funcindex{ppath\_step\_geom\_2d}.}
 \label{alg:ppath:ppath_step_geom_2d}
\end{algorithm}

Many tasks are independent of the algorithm for refraction that is
used, or if refraction is considered at all. These tasks are solved by
two functions for each atmospheric dimensionality. For 1D the
functions are \funcindex{ppath\_start\_1d} and
\funcindex{ppath\_end\_1d}, and the corresponding functions for 2D and
3D are named in the same way. The functions to calculate geometrical
path steps are denoted as \funcindex{do\_gridrange\_1d},
\funcindex{do\_gridcell\_2d} and \funcindex{do\_gridcell\_3d}. Paths
steps passing a tangent point are handled by a recursive call of the
step function. Algorithm \ref{alg:ppath:ppath_step_geom_2d} summerizes
this for geometrical 2D steps.


\section{General comments}
%===================
\label{sec:ppath:comments}

The calculation of propagation paths involves a number of mathematical
expressions and they are presented in
Sections \ref{sec:ppath:basicgeom}--\ref{sec:ppath:refreuler}. In
addition, the path calculations present a number of practical
problems. These practical problems are discussed briefly in this
section. For further details, see the code.


\subsection{Numerical precision}
%===================

The aim here is not to make a complete discussion around the limited
numerical accuracy, but just to point out some of the problems caused.
We can start by noticing that the precision with which atmospheric
positions can be given is about 0.5\,m when the numeric type is
\textindex{float} and 2\topowerten{-8}\,m for \textindex{double}
(assuming that the mantissa has 24 and 48 bits, respectively). The
numbers given correspond to the change of the position for a change of
1 bit, in either radius, latitude and longitude. Already these numbers
cause problems for the approach taken to calculate propagation paths.
For any path along the border of a grid cell, any rounding error in
the wrong direction will move the position outside the grid cell,
which would lead to a crash of the code without countermeasures.

The values above give the representation precision. The precision will
be even poorer if a position is obtained by calculations as numerical
problems tend to accumulate. The calculation precision depends on what
mathematical expressions that are involved.  For example, a radius or
length obtained by the Pythagorean relation will have a relatively
high uncertainty as the calculations involve taking the square of a
radius in the order of 6400\,km. It was found that for calculations
performed using only float as numeric type, could lead to
displacements from the true position up to 10\,m. It was first tried to
hard-code double as the numerical type for the most critical passages
of the calculations, but a total success was not achieved and some
code had to be duplicated (to be used with either the float or double
option by if-statements for the pre-compiler) to avoid compiler
warnings. A step further was then taken, and double is now hard-coded
for all internal variables of \fileindex{ppath.cc}. This deviation
from the rule to have an uniform numeric type inside ARTS was
introduced to avoid more complicated coding and it has a very small
impact on the overall calculation speed. However, this measure will
not lead to that the precision of the path calculations will be the
same for float and double, as the results will be converted to float
between each propagation path step when copied to
\builtindoc{ppath\_step}.

As pointed out above, the most critical cases are when the path goes
along the boundary of a grid cell. This situation is not common for
arbitrary observation positions, but it is a standard case for 3D
scattering calculations as the starting point for the calculations
there is always a crossing point of the atmospheric grids. The
solution to this problem is to introduce special treatment for such
geometrical paths. For strictly vertical 2D and 3D paths, the
latitude, and also longitude for 3D, of the start and end points shall
be identical. Paths in 3D with an azimuth angle of 0\degree or
180\degree\ have a constant longitude; the paths are in the north-south
plane, and this should also then be valid for the longitude value of
the start and end positions of the path step.

The variables connected to different problems associated with the
numerical inaccuracy and singularity of mathematical expressions are
defined at the top of the file \shortcode{ppath.cc}. The variables
include the accepted tolerance when making asserts in internal
functions that the given point is inside the specified grid cell.
Another example is the latitude limit to use the special mathematical
expressions needed for positions on the poles.



\subsection{Propagation paths and grid positions}
%===================

The grid positions are calculated on the same time as the path is
determined. The main reason to this is that the grid positions make it
possible to quickly determine inside which grid box the path step is
found. Without the grid positions, each call of the functions would
need a costly search to locate the starting position with respect to
the grids. If you are not familiar with grid positions, it is
recommended to read \developer, Section \ref{D-sec:interpolation}
before you continue here.

The limited numerical accuracy requires some care when setting the
grid positions. First of all, rounding errors can give a fractional
distance $< 0$ or $> 1$ and this must be avoided. The function
\funcindex{gridpos\_check\_fd} was created for this purpose, and
should be called for each grid position. This function just
sets all values below 0 to 0 and all value above 1 to 1. In addition,
the grid position for the end point of a path step (beside when there
is an intersection with the ground) must have one fractional
distance of exactly 0 or 1, but this is not ensured by
\shortcode{gridpos\_check\_fd} and for end points the function
\funcindex{gridpos\_force\_end\_fd} shall also be called.

Some care is needed to determine in which grid range a path step is
found. First of all, there exists an ambiguity for the fractional
distance at the grid points. It can either be 0 or 1. In addition, if
a position is exactly on top of a grid point, the observation
direction determines the interesting grid range. As an help to resolve
these question there is the function \funcindex{gridpos2gridrange}.
This function takes an argument describing the direction of the
line-of-sight with respect to the grids. This argument shall be set to
1 if the viewing direction is towards higher indexes. The direction
argument can be set with the following logical expressions, for the
different combinations of atmospheric dimensionality and grid of
interest:

 {\bf 1D-3D, pressure}: $\quad |\ZntAng| \leq 90\degree$

 {\bf 2D, latitude}: $\quad \ZntAng \geq 0\degree$

 {\bf 3D, latitude}: $\quad \AzmAng \leq 90\degree$

 {\bf 3D, longitude}: $\quad \AzmAng \geq 0\degree$









%%% Local Variables: 
%%% mode: latex
%%% TeX-master: "uguide"
%%% End: 

% LocalWords:  ppath cc stepGeometric stepGeometricWithLmax ppathCalc pos los
% LocalWords:  ArrayOfGridPos geom ppc geomppath gridpos fd gridrange Eq Eqs
% LocalWords:  rodgers WGS montenbruck
